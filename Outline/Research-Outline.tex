% Created 2020-08-24 Mon 15:46
% Intended LaTeX compiler: pdflatex
\documentclass[11pt]{article}
\usepackage[utf8]{inputenc}
\usepackage[T1]{fontenc}
\usepackage{graphicx}
\usepackage{grffile}
\usepackage{longtable}
\usepackage{wrapfig}
\usepackage{rotating}
\usepackage[normalem]{ulem}
\usepackage{amsmath}
\usepackage{textcomp}
\usepackage{amssymb}
\usepackage{capt-of}
\usepackage{hyperref}
\usepackage{minted}
\IfFileExists{./resources/style.sty}{\usepackage{./resources/style}}{}
\IfFileExists{./resources/referencing.sty}{\usepackage{./resources/referencing}}{}
\addbibresource{../Resources/references.bib}
\author{Ryan Greenup \& James Guerra}
\date{\today}
\title{Research Outline}
\hypersetup{
 pdfauthor={Ryan Greenup \& James Guerra},
 pdftitle={Research Outline},
 pdfkeywords={},
 pdfsubject={},
 pdfcreator={Emacs 27.1 (Org mode 9.4)}, 
 pdflang={English}}
\begin{document}

\maketitle
\tableofcontents

\begin{minted}[]{bash}
code /home/ryan/Dropbox/Studies/QuantProject/Current/Python-Quant/ & disown
\end{minted}

Here's what I gatthered from the week 3 slides

\section{Give a brief Sketch of the project}
\label{sec:org039101c}


So here is a citation \cite{lehmanReadingsMathematicsComputer2010}
I got another one here though \cite{nicodemiIntroductionAbstractAlgebra2007}
\subsection{Topic / Context}
\label{sec:orgc8ee0a5}
We are interested in the theory of problem solving, but in particular the
different approaches that can be taken to attacking a problem.

Essentially this boils down to looking at how a computer scientist and
mathematician attack a problem, although originally I thought there was no
difference, after seeing the odd way Roozbeh attacks problems I see there is a big difference.
\subsection{Motivation}
\label{sec:org9aff6e5}

\subsection{Basic Ideas}
\label{sec:orgfd2ddab}
\begin{itemize}
\item Look at FOSS CAS Systems
\begin{itemize}
\item Python (Sympy)
\item Julia
\begin{itemize}
\item Sympy integration
\item symEngine
\item Reduce.jl
\item Symata.jl
\end{itemize}
\end{itemize}

\item Maybe look at interactive sessions:
\begin{itemize}
\item Like Jupyter
\item Hydrogen
\item TeXmacs
\item org-mode?
\end{itemize}
\end{itemize}

After getting an overview of SymPy let's look at problems that are interesting (chaos, morphogenesis and order from disarray etc.)


\subsection{Where are the Mathematics}
\label{sec:org55da4e2}

\begin{itemize}
\item Trying to look at the algorithms underlying functions in Python/Sympy and other Computer algebra tools such as Maxima, Maple, Mathematica, Sage, GAP and Xcas/Giac, Yacas, Symata.jl, Reduce.jl, SymEngine.jl
\begin{itemize}
\item For Example Recursive Relations
\end{itemize}
\item Look at solving some problems related to chaos theory maybe
\begin{itemize}
\item Mandelbrot and Julia Sets
\end{itemize}
\item Look at solving some problems related to Fourier Transforms maybe
\end{itemize}


AVOID DETAILS, JUST SKETCH THE PROJECT OUT.


\subsection{Don't Forget we need a talk}
\label{sec:org87ed653}
\subsubsection{Slides In Org Mode}
\label{sec:org27f2a0a}
\begin{itemize}
\item \href{https://orgmode.org/worg/org-tutorials/non-beamer-presentations.html}{Without Beamer}
\item \href{https://orgmode.org/worg/exporters/beamer/tutorial.html}{With Beamer}
\end{itemize}

\section{What we're looking for}
\label{sec:orgbf0050f}

\begin{itemize}
\item Would a reader know what the project is about?
\item Would a reader become interested in the upcoming report?
\item Is it brief but well prepared?
\item Are the major parts or phases sketched out
\end{itemize}



\section{Download RevealJS}
\label{sec:org518cfab}
So first do \texttt{M-x package-install ox-reveal} then do \texttt{M-x load-library} and then look for \texttt{ox-reveal}

\begin{minted}[]{elisp}
(load "/home/ryan/.emacs.d/.local/straight/build/ox-reveal/ox-reveal.el")
\end{minted}

Download Reveal.js and put it in the directory as \texttt{./reveal.js}, you can do that with something like this:

\begin{minted}[]{bash}
# cd /home/ryan/Dropbox/Studies/2020Spring/QuantProject/Current/Python-Quant/Outline/
wget https://github.com/hakimel/reveal.js/archive/master.tar.gz
tar -xzvf master.tar.gz && rm master.tar.gz
mv reveal.js-master reveal.js
\end{minted}

Then just do \texttt{C-c e e R R} to export with RevealJS as opposed to PHP you won't need a fancy server, just open it in the browser.
\section{GNU Plot}
\label{sec:orged5cfe1}
\href{https://rosettacode.org/wiki/Find\_limit\_of\_recursion\#gnuplot}{limit of recursion is 250}

\begin{minted}[]{gnuplot}
complex(x,y) = x*{1,0}+y*{0,1}
mandel(x,y,z,n) = (abs(z)>2.0 || n>=200) ? \
                  n : mandel(x,y,z*z+complex(x,y),n+1)

set xrange [-1.5:0.5]
set yrange [-1:1]
set logscale z
set isosample 200
set hidden3d
set contour
splot mandel(x,y,{0,0},0) notitle
\end{minted}

\begin{center}
\includesvg[width=.9\linewidth]{one}
\end{center}


\href{http://folk.uio.no/inf3330/scripting/doc/gnuplot/Kawano/fractal/mandelbrot-e.html}{reference for image}

,\#+begin\textsubscript{src} gnuplot
\begin{minted}[]{gnuplot}

complex(x,y) = x*{1,0}+y*{0,1}
mandel(x,y,z,n) = (abs(z)>2.0 || n>=200) ? \
                  n : mandel(x,y,z*z+complex(x,y),n+1)

set xrange [-0.5:0.5]
set yrange [-0.5:0.5]
set logscale z
set isosample 100
set hidden3d
set contour
a= -0.37
b= -0.612
splot mandel(a,b,complex(x,y),0) notitle
\end{minted}

\begin{center}
\includesvg[width=.9\linewidth]{two}
\end{center}




\href{https://rosettacode.org/wiki/Mandelbrot\_set\#Python}{reference}


\begin{minted}[]{gnuplot}
rmax = 2
nmax = 100
complex (x, y) = x * {1, 0} + y * {0, 1}
mandelbrot (z, z0, n) = n == nmax || abs (z) > rmax ? n : mandelbrot (z ** 2 + z0, z0, n + 1)
set samples 200
set isosamples 200
set pm3d map
set size square
splot [-2 : .8] [-1.4 : 1.4] mandelbrot (complex (0, 0), complex (x, y), 0) notitle
\end{minted}

\begin{center}
\includesvg[width=.9\linewidth]{three}
\end{center}




\section{Heres a Gif}
\label{sec:orgb5bf6d0}


So this is a very big Gif that I'm using:

How did I make the Gif??

\url{https://dl.dropboxusercontent.com/s/rbu25urfg8sbwfu/out.gif?dl=0}
\end{document}
