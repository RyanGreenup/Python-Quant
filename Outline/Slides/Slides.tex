% Created 2020-08-26 Wed 19:12
% Intended LaTeX compiler: pdflatex
\documentclass[11pt]{article}
\usepackage[utf8]{inputenc}
\usepackage[T1]{fontenc}
\usepackage{graphicx}
\usepackage{grffile}
\usepackage{longtable}
\usepackage{wrapfig}
\usepackage{rotating}
\usepackage[normalem]{ulem}
\usepackage{amsmath}
\usepackage{textcomp}
\usepackage{amssymb}
\usepackage{capt-of}
\usepackage{hyperref}
\usepackage{minted}
\IfFileExists{../resources/style.sty}{\usepackage{../resources/style}}{}
\IfFileExists{../resources/referencing.sty}{\usepackage{../resources/referencing}}{}
\addbibresource{./bibtex-refs.bib}
\author{Ryan Greenup \& James Guerra}
\date{\today}
\title{Math with Computers}
\hypersetup{
 pdfauthor={Ryan Greenup \& James Guerra},
 pdftitle={Math with Computers},
 pdfkeywords={},
 pdfsubject={},
 pdfcreator={Emacs 27.1 (Org mode 9.4)}, 
 pdflang={English}}
\begin{document}

\maketitle
\tableofcontents

\section{Fibonacci Sequence}
\label{sec:orge7554f5}
\subsection{Computational Approach}
\label{define-the-fibonacci-numbers}
The \emph{Fibonacci} Numbers are given by:

\begin{align}
F_n = F_{n-1} + F_{n-2} \label{eq:fib-def}
\end{align}

\subsection{Defining Recursively in Python}
\label{sec:org1f81fdc}


\begin{listing}[htbp]
\begin{minted}[]{python}
  def rec_fib(k):
      if type(k) is not int:
          print("Error: Require integer values")
          return 0
      elif k == 0:
          return 0
      elif k <= 2:
          return 1
      return rec_fib(k-1) + rec_fib(k-2)
\end{minted}
\caption{\label{fib-rec-0}Defining the \emph{Fibonacci Sequence} \eqref{eq:fib-def} using Recursion}
\end{listing}

\begin{listing}[htbp]
\begin{minted}[]{python}
  start = time.time()
  rec_fib(35)
  print(str(round(time.time() - start, 3)) + "seconds")

## 2.245seconds
\end{minted}
\caption{\label{time-slow}Using the function from listing \ref{fib-rec-0} is quite slow.}
\end{listing}

\subsection{Caching to Memory}
\label{sec:orgaf410fe}

\begin{listing}[htbp]
\begin{minted}[]{python}
  from functools import lru_cache

  @lru_cache(maxsize=9999)
  def rec_fib(k):
      if type(k) is not int:
          print("Error: Require Integer Values")
          return 0
      elif k == 0:
          return 0
      elif k <= 2:
          return 1
      return rec_fib(k-1) + rec_fib(k-2)


start = time.time()
rec_fib(35)
print(str(round(time.time() - start, 3)) + "seconds")
## 0.0seconds
\end{minted}
\caption{\label{fib-cache}Caching the results of the function previously defined \ref{time-slow}}
\end{listing}

\begin{minted}[]{python}
  start = time.time()
  rec_fib(6000)
  print(str(round(time.time() - start, 9)) + "seconds")

## 8.3923e-05seconds
\end{minted}

Restructuring the problem to use iteration will allow for even greater performance as demonstrated by finding \(F_{10^{6}}\) in listing \ref{fib-iter}. Using a compiled language such as \emph{Julia} however would be thousands of times faster still, as demonstrated in listing \ref{julia-fib}.

\subsection{Solving Iteratively}
\label{sec:orgcdfe9d9}

\begin{listing}[htbp]
\begin{minted}[]{python}
  def my_it_fib(k):
      if k == 0:
          return k
      elif type(k) is not int:
          print("ERROR: Integer Required")
          return 0
      # Hence k must be a positive integer

      i  = 1
      n1 = 1
      n2 = 1

      # if k <=2:
      #     return 1

      while i < k:
         no = n1
         n1 = n2
         n2 = no + n2
         i = i + 1
      return (n1)

  start = time.time()
  my_it_fib(10**6)
  print(str(round(time.time() - start, 9)) + "seconds")

 ## 6.975890398seconds
\end{minted}
\caption{\label{fib-iter}Using Iteration to Solve the Fibonacci Sequence}
\end{listing}

\subsection{Solving With Julia is even Faster}
\label{sec:org77d6aa5}

\begin{listing}[htbp]
\begin{minted}[]{julia}
function my_it_fib(k)
    if k == 0
        return k
    elseif typeof(k) != Int
        print("ERROR: Integer Required")
        return 0
    end
    # Hence k must be a positive integer

    i  = 1
    n1 = 1
    n2 = 1

    # if k <=2:
    #     return 1
    while i < k
       no = n1
       n1 = n2
       n2 = no + n2
       i = i + 1
    end
    return (n1)
end

@time my_it_fib(10^6)

##  my_it_fib (generic function with 1 method)
##    0.000450 seconds
\end{minted}
\caption{\label{julia-fib}Using Julia with an iterative approach to solve the 1 millionth fibonacci number}
\end{listing}

In this case however an analytic solution can be found by relating discrete
mathematical problems to continuous ones as discussed below at section
.
\subsection{Exponential Generating Functions}
\label{exp-gen-func-fib-seq}
\subsubsection{Motivation}
\label{motivation}
Consider the \emph{Fibonacci Sequence} from \eqref{eq:fib-def}:


\begin{align}
    a_{n}&= a_{n - 1} + a_{n - 2} \nonumber \\
\iff a_{n+  2} &= a_{n+  1} +  a_n \label{eq:fib-def-shift}
\end{align}


from observation, this appears similar in structure to the following \emph{ordinary
differential equation}, which would be fairly easy to deal with:


\begin{align*}
f''\left( x \right)- f'\left( x \right)- f\left( x \right)=  0
\end{align*}


This would imply that \(f\left( x \right) \propto e^{mx}, \quad \exists m \in \mathbb{Z}\) because
\(\frac{\mathrm{d}\left( e^x \right) }{\mathrm{d} x} = e^x\), and so by using a power series it's quite feasable to move between discrete and continuous problems:


\begin{align*}
f\left( x \right)= e^{rx} = \sum^{\infty}_{n= 0}   \left[ r \frac{x^n}{n!} \right]
\end{align*}

\subsubsection{Example}
\label{solving-the-sequence}
Consider using the following generating function, (the derivative of the
generating function as in \eqref{eq:exp-gen-def-2} and \eqref{eq:exp-gen-def-3} is
provided in section \ref{Derivative-exp-gen-function})




\begin{alignat}{2}
    f \left( x \right) &=  \sum^{\infty}_{n= 0}   \left[ a_{n} \cdot  \frac{x^n}{n!} \right]   &= e^x \label{eq:exp-gen-def-1} \\
    f'\left( x \right) &=  \sum^{\infty}_{n= 0}   \left[ a_{n+1} \cdot  \frac{x^n}{n!} \right]  &= e^x  \label{eq:exp-gen-def-2} \\
    f''\left( x \right) &=  \sum^{\infty}_{n= 0}   \left[ a_{n+2} \cdot  \frac{x^n}{n!} \right] &= e^x  \label{eq:exp-gen-def-3}
\end{alignat}


So the recursive relation from \eqref{eq:fib-def-shift}  could be expressed :


\begin{align*}
a_{n+  2}    &= a_{n+  1} +  a_{n}\\
\frac{x^n}{n!}   a_{n+  2}    &= \frac{x^n}{n!}\left( a_{n+  1} +  a_{n}  \right)\\
\sum^{\infty}_{n= 0} \left[ \frac{x^n}{n!}   a_{n+  2} \right]        &= \sum^{\infty}_{n= 0}   \left[ \frac{x^n}{n!} a_{n+  1} \right]  + \sum^{\infty}_{n= 0}   \left[ \frac{x^n}{n!} a_{n}  \right]  \\
f''\left( x \right) &= f'\left( x \right)+  f\left( x \right)
\end{align*}


Using the theory of higher order linear differential equations with
constant coefficients it can be shown:


\begin{align*}
f\left( x \right)= c_1 \cdot  \mathrm{exp}\left[ \left( \frac{1- \sqrt{5} }{2} \right)x \right] +  c_2 \cdot  \mathrm{exp}\left[ \left( \frac{1 +  \sqrt{5} }{2} \right) \right]
\end{align*}


By equating this to the power series:


\begin{align*}
f\left( x \right)&= \sum^{\infty}_{n= 0}   \left[ \left( c_1\left( \frac{1- \sqrt{5} }{2} \right)^n +  c_2 \cdot  \left( \frac{1+ \sqrt{5} }{2} \right)^n \right) \cdot  \frac{x^n}{n} \right]
\end{align*}


Now given that:


\begin{align*}
f\left( x \right)= \sum^{\infty}_{n= 0}   \left[ a_n \frac{x^n}{n!} \right]
\end{align*}


We can conclude that:


\begin{align*}
a_n = c_1\cdot  \left( \frac{1- \sqrt{5} }{2} \right)^n +  c_2 \cdot  \left( \frac{1+  \sqrt{5} }{2} \right)
\end{align*}


By applying the initial conditions:


\begin{align*}
a_0= c_1 +  c_2  \implies  c_1= - c_2\\
a_1= c_1 \left( \frac{1+ \sqrt{5} }{2} \right) -  c_1 \frac{1-\sqrt{5} }{2}  \implies  c_1 = \frac{1}{\sqrt{5} }
\end{align*}


And so finally we have the solution to the \emph{Fibonacci Sequence} \ref{eq:fib-def-shift}:


\begin{align}
    a_n &= \frac{1}{\sqrt{5} } \left[ \left( \frac{1+  \sqrt{5} }{2}  \right)^n -  \left( \frac{1- \sqrt{5} }{2} \right)^n \right] \nonumber \\
&= \frac{\varphi^n - \psi^n}{\sqrt{5} } \nonumber\\
&=\frac{\varphi^n -  \psi^n}{\varphi - \psi} \label{eq:fib-sol}
\end{align}


where:

\begin{itemize}
\item \(\varphi = \frac{1+ \sqrt{5} }{2} \approx 1.61\ldots\)
\item \(\psi = 1-\varphi = \frac{1- \sqrt{5} }{2} \approx 0.61\ldots\)
\end{itemize}

\subsubsection{Derivative of the Exponential Generating Function}
\label{Derivative-exp-gen-function}
Differentiating the exponential generating function has the effect of shifting the sequence to the backward: \cite{lehmanReadingsMathematicsComputer2010}

\begin{align}
    f\left( x \right) &= \sum^{\infty}_{n= 0}   \left[ a_n \frac{x^n}{n!} \right] \label{eq:exp-pow-series} \\
f'\left( x \right)) &= \frac{\mathrm{d} }{\mathrm{d} x}\left( \sum^{\infty}_{n= 0}   \left[ a_n \frac{x^n}{n!} \right]  \right) \nonumber \\
&= \frac{\mathrm{d}}{\mathrm{d} x} \left( a_0 \frac{x^0}{0!} +  a_1 \frac{x^1}{1!} +  a_2 \frac{x^2}{2!}+  a_3 \frac{x^3}{3! } +  \ldots \frac{x^k}{k!} \right) \nonumber \\
&= \sum^{\infty}_{n= 0}   \left[ \frac{\mathrm{d} }{\mathrm{d} x}\left( a_n \frac{x^n}{n!} \right) \right] \nonumber \\
&= \sum^{\infty}_{n= 0}   {\left[{ \frac{a_n}{{\left({ n- 1 }\right)!}} } x^{n- 1}  \right]} \nonumber \\
\implies f'(x) &= \sum^{\infty}_{n= 0}   {\left[{ \frac{x^n}{n!}a_{n+  1} }\right]} \label{eq:exp-pow-series-sol}
\end{align}

If \(f\left( x \right)= \sum^{\infty}_{n= 0 } \left[ a_n \frac{x^n}{n!} \right]\) can it be shown by induction that \(\frac{\mathrm{d}^k }{\mathrm{d} x^k} \left(  f\left( x \right) \right)= f^{k} \left( x \right) \sum^{\infty}_{n= 0}   \left[ x^n \frac{a_{n+  k}}{n!} \right]\)

\subsubsection{Homogeneous Proof}
\label{sec:org257ef38}
An equation of the form:

\begin{align}
\sum^{\infty}_{n=0} \left[ c_{i} \cdot f^{(n)}(x) \right] = 0 \label{eq:hom-ode}
\end{align}

is said to be a homogenous linear ODE: \cite[Ch. 2]{zillDifferentialEquations2009a}

\begin{description}
\item[{Linear}] because the equation is linear with respect to \(f(x)\)
\item[{Ordinary}] because there are no partial derivatives (e.g. \(\frac{\partial }{\partial x}{\left({ f{\left({ x }\right)} }\right)}\)  )
\item[{Differential}] because the derivates of the function are concerned
\item[{Homogenous}] because the \textbf{\emph{RHS}} is 0
\begin{itemize}
\item A non-homogeous equation would have a non-zero RHS
\end{itemize}
\end{description}

There will be \(k\) solutions to a \(k^{\mathrm{th}}\) order linear ODE, each may be summed to produce a superposition which will also be a solution to the equation, \cite[Ch. 4]{zillDifferentialEquations2009a}  this will be considered as the desired complete solution (and this will be shown to be the only solution for the recurrence relation \eqref{eq:recurrence-relation-def}). These \(k\) solutions will be in one of two forms:

\begin{enumerate}
\item \(f(x)=c_{i} \cdot e^{m_{i}x}\)
\item \(f(x)=c_{i} \cdot x^{j}\cdot e^{m_{i}x}\)
\end{enumerate}

where:

\begin{itemize}
\item \(\sum^{k}_{i=0}\left[  c_{i}m^{k-i} \right] = 0\)
\begin{itemize}
\item This is referred to the characteristic equation of the recurrence relation or ODE \cite{levinSolvingRecurrenceRelations2018}
\end{itemize}
\item \(\exists i,j \in \mathbb{Z}^{+} \cap \left[0,k\right]\)
\begin{itemize}
\item These is often referred to as repeated roots \cite{levinSolvingRecurrenceRelations2018,zillMatrixExponential2009} with a multiplicity corresponding to the number of repetitions of that root \cite[\textsection 3.2]{nicodemiIntroductionAbstractAlgebra2007}
\end{itemize}
\end{itemize}

\paragraph{Unique Roots of Characteristic Equation}
\label{uniq-roots-recurrence}
\subparagraph{Example}
\label{sec:org1e2a417}
An example of a recurrence relation with all unique roots is the fibonacci sequence, as described in section \ref{solving-the-sequence}.
\subparagraph{Proof}
\label{sec:org315064d}
Consider the linear recurrence relation \eqref{eq:recurrence-relation-def}:

\begin{align}
\sum^{\infty}_{n= 0}   \left[ c_i \cdot  a_n \right] = 0, \quad \exists c \in
\mathbb{R}, \enspace \forall i<k\in\mathbb{Z}^+ \nonumber
\end{align}

By implementing the exponential generating function as shown in \eqref{eq:exp-gen-def-1}, this provides:


\begin{align}
    \sum^{k}_{i= 0}   {\left[{ c_i \cdot a_n } \right]} = 0 \nonumber \\
    \intertext{By Multiplying through and summing: } \notag \\
     \implies  \sum^{k}_{i= 0}   {\left[{ \sum^{\infty}_{n= 0}   {\left[{ c_i a_n \frac{x^n}{n!} }\right]}  }\right]}  \nonumber = 0 \\
     \sum^{k}_{i= 0}    {\left[{ c_i \sum^{\infty}_{n= 0}   {\left[{  a_n \frac{x^n}{n!} }\right]}  }\right]}  \nonumber = 0 \\
\end{align}

Recall from \eqref{eq:exp-gen-def-1} the generating function \(f{\left({ x }\right)}\):

\begin{align}
\sum^{k}_{i= 0}   {\left[{ c_i f^{{\left({ k }\right)} } } {\left({ x }\right)} \right]} \label{eq:exp-gen-def-proof}  &= 0
\end{align}


Now assume that the solution exists and all roots of the characteristic polynomial are unique (i.e. the solution is of the form \(f{\left({ x }\right)} \propto e^{m_i x}: \quad m_i \neq m_j \forall i\neq j\)), this implies that  \cite[Ch. 4]{zillDifferentialEquations2009a} :

\begin{align}
    f{\left({ x }\right)} = \sum^{k}_{i= 0}   {\left[{ k_i e^{m_i x} }\right]}, \quad \exists m,k \in \mathbb{C} \nonumber
\end{align}

This can be re-expressed in terms of the exponential power series, in order to relate the solution of the function \(f{\left({ x }\right)}\) back to a solution of the sequence \(a_n\), (see section for a derivation of the exponential power series):

\begin{align}
    \sum^{k}_{i= 0}   {\left[{ k_i e^{m_i x}  }\right]}  &= \sum^{k}_{i= 0}   {\left[{ k_i \sum^{\infty}_{n= 0}   \frac{{\left({ m_i x }\right)}^n}{n!}  }\right]}  \nonumber \\
							 &= \sum^{k}_{i= 0}  \sum^{\infty}_{n= 0}   k_i m_i^n \frac{x^n}{n!} \nonumber\\
							 &=    \sum^{\infty}_{n= 0} \sum^{k}_{i= 0}   k_i m_i^n \frac{x^n}{n!} \nonumber \\
							 &= \sum^{\infty}_{n= 0} {\left[{ \frac{x^n}{n!}  \sum^{k}_{i=0}   {\left[{ k_im^n_i }\right]}  }\right]}, \quad \exists k_i \in \mathbb{C}, \enspace \forall i \in \mathbb{Z}^+\cap {\left[{ 1, k }\right]}     \label{eq:unique-root-sol-power-series-form}
\end{align}

Recall the definition of the generating function from \ref{eq:exp-gen-def-proof}, by relating this to \eqref{eq:unique-root-sol-power-series-form}:

\begin{align}
    f{\left({ x }\right)} &= \sum^{\infty}_{n= 0}   {\left[{  \frac{x^n}{n!} a_n }\right]} \nonumber \\
&= \sum^{\infty}_{n= 0} {\left[{ \frac{x^n}{n!}  \sum^{k}_{i=0}   {\left[{ k_im^n_i }\right]}  }\right]}  \nonumber \\
      \implies  a_n &= \sum^{k}_{n= 0} {\left[{ k_im_i^n }\right]}     \nonumber \\ \nonumber
\square
\end{align}

This can be verified by the fibonacci sequence as shown in section \ref{solving-the-sequence}, the solution to the characteristic equation is \(m_1 = \varphi, m_2 = {\left({ 1-\varphi }\right)}\) and the corresponding solution to the linear ODE and recursive relation are:

\begin{alignat}{4}
    f{\left({ x }\right)} &= &c_1 e^{\varphi x} +  &c_2 e^{{\left({ 1-\varphi }\right)} x}, \quad &\exists c_1, c_2 \in \mathbb{R} \subset \mathbb{C} \nonumber \\
    \iff  a_n &= &k_1 n^{\varphi} +  &k_2 n^{1- \varphi}, &\exists k_1, k_2 \in \mathbb{R} \subset \mathbb{C} \nonumber
\end{alignat}

\paragraph{Repeated Roots of Characteristic Equation}
\label{rep-roots-recurrence}
\subparagraph{Example}
\label{sec:orgedf9608}
Consider the following recurrence relation:

\begin{align}
    a_n -  10a_{n+ 1} +  25a_{n+  2}&= 0 \label{eq:hom-repeated-roots-recurrence} \\
    \implies  \sum^{\infty}_{n= 0}   {\left[{ a_n \frac{x^n}{n!} }\right]} - 10 \sum^{\infty}_{n= 0}   {\left[{ \frac{x^n}{n!}+    }\right]} + 25 \sum^{\infty}_{n= 0 }   {\left[{  a_{n+  2 }\frac{x^n}{n!} }\right]}&= 0 \nonumber
\end{align}

By applying the definition of the exponential generating function at \eqref{eq:exp-gen-def-1} :

\begin{align}
    f''{\left({ x }\right)}- 10f'{\left({ x }\right)}+  25f{\left({ x }\right)}= 0 \nonumber \label{eq:rep-roots-func-ode}
\end{align}

By implementing the already well-established theory of linear ODE's, the characteristic equation for \eqref{eq:rep-roots-func-ode} can be expressed as:

\begin{align}
    m^2- 10m+  25 = 0 \nonumber \\
    {\left({ m- 5 }\right)}^2 = 0 \nonumber \\
    m= 5 \label{eq:rep-roots-recurrence-char-sol}
\end{align}

Herein lies a complexity, in order to solve this, the solution produced from \eqref{eq:rep-roots-recurrence-char-sol} can be used with the \emph{Reduction of Order} technique to produce a solution that will be of the form \cite[\textsection 4.3]{zillMatrixExponential2009}.

\begin{align}
    f{\left({ x }\right)}= c_1e^{5x} +  c_2 x e^{5x} \label{eq:rep-roots-ode-sol}
\end{align}

\eqref{eq:rep-roots-ode-sol} can be expressed in terms of the exponential power series in order to try and relate the solution for the function back to the generating function,
observe however the following power series identity (TODO Prove this in section ):

\begin{align}
    x^ke^x &= \sum^{\infty}_{n= 0}   {\left[{ \frac{x^n}{{\left({ n- k }\right)}!} }\right]}, \quad \exists k \in \mathbb{Z}^+ \label{eq:uniq-roots-pow-series-ident}
\end{align}

by applying identity \eqref{eq:uniq-roots-pow-series-ident} to equation \eqref{eq:rep-roots-ode-sol}

\begin{align}
    \implies  f{\left({ x }\right)} &= \sum^{\infty}_{n= 0}   {\left[{ c_1 \frac{{\left({ 5x }\right)}^n}{n!} }\right]}  +  \sum^{\infty}_{n= 0}   {\left[{ c_2 n \frac{{\left({ 5x^n }\right)}}{n{\left({ n-1 }\right)}!} }\right]} \nonumber \\
 &= \sum^{\infty}_{n= 0}   {\left[{ \frac{x^n}{n!} {\left({ c_{1}5^n +  c_2 n 5^n   }\right)} }\right]} \nonumber
\end{align}

Given the defenition of the exponential generating function from \eqref{eq:exp-gen-def-1}

\begin{align}
    f{\left({ x }\right)}&=     \sum^{\infty}_{n= 0}   {\left[{ a_n \frac{x^n}{n!} }\right]} \nonumber \\
    \iff a_n &= c_{15}^n +  c_2n_5^n \nonumber \\ \nonumber
    \ \nonumber \\
    \square \nonumber
\end{align}
\subparagraph{Generalised Example}
\label{sec:org6837557}

\subparagraph{Proof}
\label{sec:orgb5bdaa9}
In order to prove the the solution for a \(k^{\mathrm{th}}\) order recurrence relation with \(k\) repeated


Consider a recurrence relation of the form:

\begin{align}
     \sum^{k}_{n= 0}   {\left[{ c_i a_n }\right]}  = 0 \nonumber \\
      \implies  \sum^{\infty}_{n= 0}   \sum^{k}_{i= 0}   c_i a_n \frac{x^n}{n!} = 0 \nonumber \\
      \sum^{k}_{i= 0}   \sum^{\infty}_{n= 0}   c_i a_n \frac{x^n}{n!} \nonumber
\end{align}

By substituting for the value of the generating function (from \eqref{eq:exp-gen-def-1}):

\begin{align}
    \sum^{k}_{i= 0}   {\left[{ c_if^{{\left({ k }\right)}}  {\left({ x }\right)}    }\right]} \label{eq:gen-form-rep-roots-ode}
\end{align}

Assume that \eqref{eq:gen-form-rep-roots-ode} corresponds to a charecteristic polynomial with only 1 root of multiplicity \(k\), the solution would hence be of the form:

\begin{align}
			 & \sum^{k}_{i= 0}   {\left[{ c_i m^i }\right]} = 0 \wedge m=B, \enspace  \exists! B \in \mathbb{C} \nonumber \\
 \implies      f{\left({ x }\right)}&= \sum^{k}_{i= 0}   {\left[{ x^i A_i e^{mx} }\right]}, \quad \exists A \in \mathbb{C}^+, \enspace \forall i \in {\left[{ 1,k }\right]} \cap \mathbb{N}  \label{eq:sol-rep-roots-ode} \\
\end{align}

Recall the following power series identity (proved in section xxx):

\begin{align}
x^k e^x = \sum^{\infty}_{n= 0} {\left[{ \frac{x^n}{{\left({ n- k }\right)}!} }\right]}     \nonumber
\end{align}

By applying this to \eqref{eq:sol-rep-roots-ode} :

\begin{align}
f{\left({ x }\right)}&=     \sum^{k}_{i= 0}   {\left[{ A_i \sum^{\infty}_{n= 0}   {\left[{ \frac{{\left({ x m }\right)}^n}{{\left({ n- i }\right)}!} }\right]}  }\right]} \nonumber \\
&=     \sum^{\infty}_{n= 0}   {\left[{ \sum^{k}_{i=0} {\left[{ \frac{x^n}{n!}  \frac{n!}{{\left({ n- i }\right)}} A_i m^n }\right]}       }\right]} # \\
&=     \sum^{\infty}_{n= 0} {\left[{ \frac{x^n}{n!}   \sum^{k}_{i=0} {\left[{  \frac{n!}{{\left({ n- i }\right)}} A_i m^n }\right]}       }\right]}
\end{align}

Recall the generating function that was used to get \ref{eq:gen-form-rep-roots-ode}:

\begin{align}
f{\left({ x }\right)}&= \sum^{\infty}_{n= 0}   {\left[{ a_n \frac{x^n}{n!} }\right]}      \nonumber \\
 \implies  a_n &= \sum^{k}_{i= 0}   {\left[{ A_i \frac{n!}{{\left({ n- i }\right)}!} m^n  }\right]} \nonumber \\
 &= \sum^{k}_{i= 0}   {\left[{ m^n A_i \prod_{0}^{k} {\left[{ n- {\left({ i- 1 }\right)} }\right]}   }\right]}
& \intertext{$\because \enspace i \leq k$} \notag \\
 &= \sum^{k}_{i= 0} {\left[{ A_i^* m^n n^i }\right]}, \quad \exists A_i \in \mathbb{C}, \enspace \forall i\leqk \in \mathbb{Z}^+ \nonumber \\
\ \nonumber \\
\square \nonumber
\end{align}



\paragraph{General Proof}
\label{sec:orgb9676d4}
In sections \ref{uniq-roots-recurrence} and \ref{uniq-roots-recurrence} it was shown that a recurrence relation can be related to an ODE and then that solution can be transformed to provide a solution for the recurrence relation, when the charecteristic polynomial has either complex roots or 1 repeated root. Generally the solution to a linear ODE will be a superposition of solutions for each root, repeated or unique and so here it will be shown that these two can be combined and that the solution will still hold.

Consider a Recursive relation with constant coefficients:

$$
\sum^{\infty}_{n= 0}   \left[ c_i \cdot  a_n \right] = 0, \quad \exists c \in
\mathbb{R}, \enspace \forall i<k\in\mathbb{Z}^+
$$

This can be expressed in terms of the exponential generating function:

$$
\sum^{\infty}_{n= 0}   \left[ c_i \cdot  a_n \right] = 0\\
\implies \sum^{\infty}_{n= 0}   \left[\sum^{\infty}_{n= 0}   \left[ c_i \cdot
a_n  \right]   \right] = 0
$$

\begin{itemize}
\item Use the Generating function to get an ODE
\item The ODE will have a solution that is a combination of the above two forms
\item The solution will translate back to a combination of both above forms
\end{itemize}

\subsection{Fibonacci Sequence and the Golden Ratio}
\label{fib-golden-ratio-proof}
The \emph{Fibonacci Sequence} is actually very interesting, observe that the ratios of the terms converge to the \emph{Golden Ratio}:

\begin{align*}
    F_n &= \frac{\varphi^n-\psi^n}{\varphi-\psi} = \frac{\varphi^n-\psi^n}{\sqrt 5} \\
    \iff \frac{F_{n+1}}{F_n}	&= \frac{\varphi^{n+ 1} - \psi^{n+  1}}{\varphi^{n} - \psi^{n}} \\
    \iff \lim_{n \rightarrow \infty}\left[ \frac{F_{n+1}}{F_n} \right]	&= \lim_{n \rightarrow \infty}\left[ \frac{\varphi^{n+ 1} - \psi^{n+  1}}{\varphi^{n} - \psi^{n}} \right] \\
&= \frac{\varphi^{n+ 1} -\lim_{n \rightarrow \infty}\left[ \psi^{n +  1} \right] }{\varphi^{n} - \lim_{n \rightarrow \infty}\left[ \psi^n \right] } \\
\text{because $\mid \psi \mid < 0$ $n \rightarrow \infty \implies \psi^{n} \rightarrow 0$:} \\
&= \frac{\varphi^{n+  1} -  0}{\varphi^{n} -  0} \\
&= \varphi
\end{align*}

We'll come back to this later on when looking at spirals and fractals.

This can also be shown by using analysis, let \$L=\(\lim\)\textsubscript{n \(\rightarrow\) \(\infty\)} \left[ \frac\{F\textsubscript{n+1}\}\{F\textsubscript{n}\} \right], then:
\end{document}
