%% ! TEX root = /home/ryan/Dropbox/Studies/MathModelling/Notes/Diffeq/latex_working/00_master.tex
% Copy realpath with Spc-f-y

\documentclass[12 pt]{article}
\usepackage{/home/ryan/Dropbox/profiles/Templates/LaTeX/ScreenStyle}

%\usepackage{~/Dropbox/profiles/Templates/LaTeX/ScreenStyle}
%LaTeX doesn't work with the ~


%%%%%%%Title
\title{\color{coltit} \Huge Recursive Relation and ODEs}
\author{Ryan Greenup ; 1780-5315}



%%%%%%%%%%%%%%%%%%%%%4%%%%%%%%%
%%%%%%%Document%%%%%%%%%%%%%%%%
%%%%%%%%%%%%%%%%%%%%%%%% %%%%%%%
\begin{document}

\maketitle
\tableofcontents

\begin{figure}[h!]
\begin{tikzpicture}[domain = 0:10, scale = (2/3)]
  \clip (-1,-1) rectangle (12,12);
  \draw[->, thick] (0,0) -- (0,10) node[right] {$C$ {\scriptsize nmol $\cdot  $ L$^{-1}$}};
  \draw[->, thick] (0,0) -- (10, 0) node[right] {$t$ };
  \draw[] [out=270, in = 180]  (0,5) to (3,2) node[right] {\scriptsize $\left( t_{\textit{min}}, c_{\textit{min}} \right)$};
  \draw[dashed] (3,2)--(3,5);
  \draw[] [out=270, in=180] (3,5) to (6,3) ;
  \draw[dashed] (6,3)--(6,6);
  \draw[] [out=270, in=180] (6,6) to (9,4) ;
  \draw[dashed] (9,4)--(9, 7);
  \draw[ ] [out=270, in=180] (9,7) to (12,4);
  \draw[dotted]  (0,7)--(12,7) node[below left] {$C_n = H$};
  \draw[dotted]  (0,4)--(12,4) node[below left] {$R_\infty = L$};
  \node [left] at (0,5) {$C_0$};
  \draw[<-> ] (0.4,4)--(0.4,7) node[below right] {\tiny $C_0 = H-L$};
\end{tikzpicture}
  \caption{Diagram of Blood Levels over time}
  \label{fig:Concentration Plot}
\end{figure}

Consider a Recursive relation with constant coefficients:

$$
\sum^{\infty}_{n= 0}   \left[ c_i \cdot  a_n \right] = 0, \quad \exists c \in
\mathbb{R}, \enspace \forall i<k\in\mathbb{Z}^+
$$

This can be expressed in terms of the exponential generating function $f\left( x \right)$ from \eqref{expDefFunc}:

$$\begin{aligned}
    \sum^{k}_{i= 0}   \left[ c_i \cdot  a_{n+i} \right] = 0 \\
\implies \sum^{k}_{i= 0}   \left[\sum^{\infty}_{n= 0}   \left[ c_i \cdot
a_{n+i} \frac{x_n}{n!}  \right]   \right] = 0
\end{aligned}$$

From \eqref{eq:diffExpGen} it is known that $\frac{\mathrm{d} }{\mathrm{d} x}\left( f\left( x \right) \right)= \frac{\mathrm{d} }{\mathrm{d} x}\left( \sum^{\infty}_{n= 0}   \left[ a_n \frac{x^n}{n!} \right]  \right) = \sum^{\infty}_{n= 0} \left[ a_{n+  1} \frac{x_n}{n!} \right]$ and hence:

$$\begin{aligned}
\sum^{k}_{i= 0}   \left[\sum^{\infty}_{n= 0}   \left[ c_i \cdot
a_{n+i} \frac{x_n}{n!}  \right]   \right] = 0 \\
 \implies  \sum^{k}_{n= 0}   \left[c_i  \cdot   f^{\left( k \right)}\left( x \right) \right] &=  0 \\
\end{aligned}$$

This is a homogenous $k^{\mathrm{th}}$ order linear ODE with constant coefficients, assume that all solutions exist, there will be $k$ solutions either of the form $e^{mx}$ or $x^k \cdot  e^x$  will be such that 

 
\end{document}



%%% Local Variables: 
%%% mode: latex
%%% TeX-master: "/home/ryan/Dropbox/Studies/MathModelling/Notes/Diffeq/latex_working/00_master.tex"
%%% End: 











