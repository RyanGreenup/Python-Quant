% Created 2020-09-16 Wed 13:41
% Intended LaTeX compiler: pdflatex
\documentclass[11pt]{article}
\usepackage[utf8]{inputenc}
\usepackage[T1]{fontenc}
\usepackage{graphicx}
\usepackage{grffile}
\usepackage{longtable}
\usepackage{wrapfig}
\usepackage{rotating}
\usepackage[normalem]{ulem}
\usepackage{amsmath}
\usepackage{textcomp}
\usepackage{amssymb}
\usepackage{capt-of}
\usepackage{hyperref}
\usepackage{listings}
\IfFileExists{../resources/style.sty}{\usepackage{../resources/style}}{}
\IfFileExists{../resources/referencing.sty}{\usepackage{../resources/referencing}}{}
\addbibresource{./bibtex-refs.bib}
\usepackage{svg}
\usepackage{tikz}
\author{Ryan Greenup \& James Guerra}
\date{\today}
\title{Fractal Dimensions}
\hypersetup{
 pdfauthor={Ryan Greenup \& James Guerra},
 pdftitle={Fractal Dimensions},
 pdfkeywords={},
 pdfsubject={},
 pdfcreator={Emacs 27.1 (Org mode 9.4)}, 
 pdflang={English}}
\begin{document}

\maketitle
\tableofcontents


\section{Backlinks}
\label{sec:org337658a}
\href{../../README.org}{Fractal Dimensions}

\section{Introduction}
\label{sec:org0844f86}
The concept of dimension is typically used to describe the number of
perpendicular edges an object has or the number of non-interacting features in a
data set.

This concept can be expanded however,

\begin{center}
\includesvg[width=.9\linewidth]{square}
\end{center}
\end{document}
