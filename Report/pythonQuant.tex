% Created 2020-08-12 Wed 11:15
% Intended LaTeX compiler: pdflatex
\documentclass[11pt]{article}
\usepackage[utf8]{inputenc}
\usepackage[T1]{fontenc}
\usepackage{graphicx}
\usepackage{grffile}
\usepackage{longtable}
\usepackage{wrapfig}
\usepackage{rotating}
\usepackage[normalem]{ulem}
\usepackage{amsmath}
\usepackage{textcomp}
\usepackage{amssymb}
\usepackage{capt-of}
\usepackage{hyperref}
\usepackage{minted}
\usepackage[citestyle=numeric, bibstyle=numeric,hyperref=true,backref=true, maxcitenames=3,url=true,backend=biber,natbib=true]{biblatex}
\usepackage{style}
\addbibresource{../Resources/references.bib}
\author{Ryan Greenup \& James Guerra}
\date{\today}
\title{Python Quantitative Project}
\hypersetup{
 pdfauthor={Ryan Greenup \& James Guerra},
 pdftitle={Python Quantitative Project},
 pdfkeywords={},
 pdfsubject={},
 pdfcreator={Emacs 26.3 (Org mode 9.4)}, 
 pdflang={English}}
\begin{document}

\maketitle
\tableofcontents

\begin{minted}[]{bash}
code /home/ryan/Dropbox/Studies/QuantProject/Current/Python-Quant/ & disown
\end{minted}

\section[What the unit involves.]{What the unit involves.\hfill{}\textsc{Ryan:James}}
\label{sec:orge9798b6}
Writing and presenting.
\section{Python}
\label{sec:org0e3c86d}
Two parts:
\begin{enumerate}
\item Figure out the python
\item IMPORTANT: MUST have new math
\begin{enumerate}
\item Dr. Hazrat may have some new math to use.
\end{enumerate}
\end{enumerate}

\section[Matrix Exponentiation]{Matrix Exponentiation\hfill{}\textsc{Ryan}}
\label{sec:orgc027797}
\subsubsection{Implementation in Sympy}
\label{sec:org55a0578}
The Matrix Exponential is implemented in areas of:

\begin{itemize}
\item Graph Centrality modelling \cite{parkPowerWalkRevisiting2013}
\item Systems of Linear Differential Equations \cite[Ch. 8.4]{Zil2009}
\item Theory of Algebraic Lie Groups \cite[Ch. 2]{hallLieGroupsLie2015}
\end{itemize}

However the method to implement matrix exponentiation provided \href{https://docs.sympy.org/latest/tutorial/matrices.html}{by the documentation} \cite{MatricesSymPyDocumentation2020} and \href{https://github.com/sympy/sympy/issues/6218}{referenced in the development repository} \cite{MatrixExponentialIssue2019} does not appear to be implemented very well, for example the following provides a very long result:

\begin{minted}[]{python}
  from __future__ import division
  from sympy import *
  x, y, z, t = symbols('x y z t')
  k, m, n = symbols('k m n', integer=True)
  f, g, h = symbols('f g h', cls=Function)
  init_printing()
  init_printing(use_latex='mathjax', latex_mode='equation')


  import pyperclip
  def lx(expr):
      pyperclip.copy(latex(expr))
      print(expr)
\end{minted}

\begin{minted}[]{python}
A = Matrix([
    [11, 12, 13],
    [21, 22, 23],
    [31, 32, 33]
])

  expr = exp(A)
  expr.doit()
\end{minted}

$$
{
\scriptsize
\left[\begin{matrix}- \frac{1}{- \frac{33}{94} + \frac{5 \sqrt{1149}}{94}}
\left(\frac{- \frac{6552}{- \sqrt{1149} - 22} + 552}{\left(- \sqrt{1149} -
22\right) \left(- \frac{59 \sqrt{1149}}{95} - \frac{1837}{95}\right)} -
\frac{26}{- \sqrt{1149} - 22}\right) - \frac{1}{\frac{33 \sqrt{1149}}{2303} +
\frac{5745}{2303}} \left(-1 - \frac{1}{- \frac{33}{94} + \frac{5
\sqrt{1149}}{94}} \left(\frac{26}{- \sqrt{1149} - 22} - \frac{- \frac{6552}{-
\sqrt{1149} - 22} + 552}{\left(- \sqrt{1149} - 22\right) \left(- \frac{59
\sqrt{1149}}{95} - \frac{1837}{95}\right)} + 2\right)\right) \left(-
\frac{13}{-22 + \sqrt{1149}} + \frac{- \frac{3276}{-22 + \sqrt{1149}} +
276}{\left(-22 + \sqrt{1149}\right) \left(- \frac{1837}{95} + \frac{59
\sqrt{1149}}{95}\right)} - \frac{1}{- \frac{33}{94} + \frac{5 \sqrt{1149}}{94}}
\left(\frac{- \frac{3276}{- \sqrt{1149} - 22} + 276}{\left(- \sqrt{1149} -
22\right) \left(- \frac{59 \sqrt{1149}}{95} - \frac{1837}{95}\right)} -
\frac{13}{- \sqrt{1149} - 22}\right) \left(- \frac{26}{-22 + \sqrt{1149}} -
\frac{- \frac{273}{-22 + \sqrt{1149}} + 23}{- \frac{1837}{95} + \frac{59
\sqrt{1149}}{95}} + \frac{- \frac{6552}{-22 + \sqrt{1149}} + 552}{\left(-22 +
\sqrt{1149}\right) \left(- \frac{1837}{95} + \frac{59
\sqrt{1149}}{95}\right)}\right)\right) + \frac{1}{\left(\frac{33
\sqrt{1149}}{2303} + \frac{5745}{2303}\right) e^{-33 + \sqrt{1149}}} \left(-1 -
\frac{1}{- \frac{33}{94} + \frac{5 \sqrt{1149}}{94}} \left(\frac{26}{-
\sqrt{1149} - 22} - \frac{- \frac{6552}{- \sqrt{1149} - 22} + 552}{\left(-
\sqrt{1149} - 22\right) \left(- \frac{59 \sqrt{1149}}{95} -
\frac{1837}{95}\right)} + 2\right)\right) \left(- \frac{13}{-22 + \sqrt{1149}} +
\frac{- \frac{3276}{-22 + \sqrt{1149}} + 276}{\left(-22 + \sqrt{1149}\right)
\left(- \frac{1837}{95} + \frac{59 \sqrt{1149}}{95}\right)}\right) + 1 +
\left(\frac{- \frac{3276}{- \sqrt{1149} - 22} + 276}{\left(- \sqrt{1149} -
22\right) \left(- \frac{59 \sqrt{1149}}{95} - \frac{1837}{95}\right)} -
\frac{13}{- \sqrt{1149} - 22}\right) \left(- \frac{1}{\left(- \frac{33}{94} +
\frac{5 \sqrt{1149}}{94}\right) \left(\frac{33 \sqrt{1149}}{2303} +
\frac{5745}{2303}\right)} \left(-1 - \frac{1}{- \frac{33}{94} + \frac{5
\sqrt{1149}}{94}} \left(\frac{26}{- \sqrt{1149} - 22} - \frac{- \frac{6552}{-
\sqrt{1149} - 22} + 552}{\left(- \sqrt{1149} - 22\right) \left(- \frac{59
\sqrt{1149}}{95} - \frac{1837}{95}\right)} + 2\right)\right) \left(-
\frac{26}{-22 + \sqrt{1149}} - \frac{- \frac{273}{-22 + \sqrt{1149}} + 23}{-
\frac{1837}{95} + \frac{59 \sqrt{1149}}{95}} + \frac{- \frac{6552}{-22 +
\sqrt{1149}} + 552}{\left(-22 + \sqrt{1149}\right) \left(- \frac{1837}{95} +
\frac{59 \sqrt{1149}}{95}\right)}\right) + \frac{2}{- \frac{33}{94} + \frac{5
\sqrt{1149}}{94}}\right) e^{33 + \sqrt{1149}} & - \frac{1}{- \frac{33}{94} +
\frac{5 \sqrt{1149}}{94}} \left(\frac{- \frac{3276}{- \sqrt{1149} - 22} +
276}{\left(- \sqrt{1149} - 22\right) \left(- \frac{59 \sqrt{1149}}{95} -
\frac{1837}{95}\right)} - \frac{13}{- \sqrt{1149} - 22}\right) +
\frac{1}{\left(- \frac{33}{94} + \frac{5 \sqrt{1149}}{94}\right) \left(\frac{33
\sqrt{1149}}{2303} + \frac{5745}{2303}\right)} \left(- \frac{13}{-22 +
\sqrt{1149}} + \frac{- \frac{3276}{-22 + \sqrt{1149}} + 276}{\left(-22 +
\sqrt{1149}\right) \left(- \frac{1837}{95} + \frac{59 \sqrt{1149}}{95}\right)} -
\frac{1}{- \frac{33}{94} + \frac{5 \sqrt{1149}}{94}} \left(\frac{- \frac{3276}{-
\sqrt{1149} - 22} + 276}{\left(- \sqrt{1149} - 22\right) \left(- \frac{59
\sqrt{1149}}{95} - \frac{1837}{95}\right)} - \frac{13}{- \sqrt{1149} -
22}\right) \left(- \frac{26}{-22 + \sqrt{1149}} - \frac{- \frac{273}{-22 +
\sqrt{1149}} + 23}{- \frac{1837}{95} + \frac{59 \sqrt{1149}}{95}} + \frac{-
\frac{6552}{-22 + \sqrt{1149}} + 552}{\left(-22 + \sqrt{1149}\right) \left(-
\frac{1837}{95} + \frac{59 \sqrt{1149}}{95}\right)}\right)\right)
\left(\frac{13}{- \sqrt{1149} - 22} - \frac{- \frac{3276}{- \sqrt{1149} - 22} +
276}{\left(- \sqrt{1149} - 22\right) \left(- \frac{59 \sqrt{1149}}{95} -
\frac{1837}{95}\right)} + 1\right) - \frac{1}{\left(- \frac{33}{94} + \frac{5
\sqrt{1149}}{94}\right) \left(\frac{33 \sqrt{1149}}{2303} +
\frac{5745}{2303}\right) e^{-33 + \sqrt{1149}}} \left(- \frac{13}{-22 +
\sqrt{1149}} + \frac{- \frac{3276}{-22 + \sqrt{1149}} + 276}{\left(-22 +
\sqrt{1149}\right) \left(- \frac{1837}{95} + \frac{59
\sqrt{1149}}{95}\right)}\right) \left(\frac{13}{- \sqrt{1149} - 22} - \frac{-
\frac{3276}{- \sqrt{1149} - 22} + 276}{\left(- \sqrt{1149} - 22\right) \left(-
\frac{59 \sqrt{1149}}{95} - \frac{1837}{95}\right)} + 1\right) + \left(\frac{-
\frac{3276}{- \sqrt{1149} - 22} + 276}{\left(- \sqrt{1149} - 22\right) \left(-
\frac{59 \sqrt{1149}}{95} - \frac{1837}{95}\right)} - \frac{13}{- \sqrt{1149} -
22}\right) \left(\frac{1}{\left(- \frac{33}{94} + \frac{5
\sqrt{1149}}{94}\right)^{2} \left(\frac{33 \sqrt{1149}}{2303} +
\frac{5745}{2303}\right)} \left(- \frac{26}{-22 + \sqrt{1149}} - \frac{-
\frac{273}{-22 + \sqrt{1149}} + 23}{- \frac{1837}{95} + \frac{59
\sqrt{1149}}{95}} + \frac{- \frac{6552}{-22 + \sqrt{1149}} + 552}{\left(-22 +
\sqrt{1149}\right) \left(- \frac{1837}{95} + \frac{59
\sqrt{1149}}{95}\right)}\right) \left(\frac{13}{- \sqrt{1149} - 22} - \frac{-
\frac{3276}{- \sqrt{1149} - 22} + 276}{\left(- \sqrt{1149} - 22\right) \left(-
\frac{59 \sqrt{1149}}{95} - \frac{1837}{95}\right)} + 1\right) + \frac{1}{-
\frac{33}{94} + \frac{5 \sqrt{1149}}{94}}\right) e^{33 + \sqrt{1149}} &
\frac{1}{\left(\frac{33 \sqrt{1149}}{2303} + \frac{5745}{2303}\right) e^{-33 +
\sqrt{1149}}} \left(- \frac{13}{-22 + \sqrt{1149}} + \frac{- \frac{3276}{-22 +
\sqrt{1149}} + 276}{\left(-22 + \sqrt{1149}\right) \left(- \frac{1837}{95} +
\frac{59 \sqrt{1149}}{95}\right)}\right) - \frac{1}{\frac{33
\sqrt{1149}}{2303} + \frac{5745}{2303}} \left(- \frac{13}{-22 + \sqrt{1149}} +
\frac{- \frac{3276}{-22 + \sqrt{1149}} + 276}{\left(-22 + \sqrt{1149}\right)
\left(- \frac{1837}{95} + \frac{59 \sqrt{1149}}{95}\right)} - \frac{1}{-
\frac{33}{94} + \frac{5 \sqrt{1149}}{94}} \left(\frac{- \frac{3276}{-
\sqrt{1149} - 22} + 276}{\left(- \sqrt{1149} - 22\right) \left(- \frac{59
\sqrt{1149}}{95} - \frac{1837}{95}\right)} - \frac{13}{- \sqrt{1149} -
22}\right) \left(- \frac{26}{-22 + \sqrt{1149}} - \frac{- \frac{273}{-22 +
\sqrt{1149}} + 23}{- \frac{1837}{95} + \frac{59 \sqrt{1149}}{95}} + \frac{-
\frac{6552}{-22 + \sqrt{1149}} + 552}{\left(-22 + \sqrt{1149}\right) \left(-
\frac{1837}{95} + \frac{59 \sqrt{1149}}{95}\right)}\right)\right) -
\frac{e^{33 + \sqrt{1149}}}{\left(- \frac{33}{94} + \frac{5
\sqrt{1149}}{94}\right) \left(\frac{33 \sqrt{1149}}{2303} +
\frac{5745}{2303}\right)} \left(\frac{- \frac{3276}{- \sqrt{1149} - 22} +
276}{\left(- \sqrt{1149} - 22\right) \left(- \frac{59 \sqrt{1149}}{95} -
\frac{1837}{95}\right)} - \frac{13}{- \sqrt{1149} - 22}\right) \left(-
\frac{26}{-22 + \sqrt{1149}} - \frac{- \frac{273}{-22 + \sqrt{1149}} + 23}{-
\frac{1837}{95} + \frac{59 \sqrt{1149}}{95}} + \frac{- \frac{6552}{-22 +
\sqrt{1149}} + 552}{\left(-22 + \sqrt{1149}\right) \left(- \frac{1837}{95} +
\frac{59 \sqrt{1149}}{95}\right)}\right)\\-2 - \frac{1}{\left(-
\frac{1837}{95} + \frac{59 \sqrt{1149}}{95}\right) \left(\frac{33
\sqrt{1149}}{2303} + \frac{5745}{2303}\right) e^{-33 + \sqrt{1149}}} \left(-1 -
\frac{1}{- \frac{33}{94} + \frac{5 \sqrt{1149}}{94}} \left(\frac{26}{-
\sqrt{1149} - 22} - \frac{- \frac{6552}{- \sqrt{1149} - 22} + 552}{\left(-
\sqrt{1149} - 22\right) \left(- \frac{59 \sqrt{1149}}{95} -
\frac{1837}{95}\right)} + 2\right)\right) \left(- \frac{273}{-22 +
\sqrt{1149}} + 23\right) + \frac{2}{\frac{33 \sqrt{1149}}{2303} +
\frac{5745}{2303}} \left(-1 - \frac{1}{- \frac{33}{94} + \frac{5
\sqrt{1149}}{94}} \left(\frac{26}{- \sqrt{1149} - 22} - \frac{- \frac{6552}{-
\sqrt{1149} - 22} + 552}{\left(- \sqrt{1149} - 22\right) \left(- \frac{59
\sqrt{1149}}{95} - \frac{1837}{95}\right)} + 2\right)\right) \left(-
\frac{13}{-22 + \sqrt{1149}} + \frac{- \frac{3276}{-22 + \sqrt{1149}} +
276}{\left(-22 + \sqrt{1149}\right) \left(- \frac{1837}{95} + \frac{59
\sqrt{1149}}{95}\right)} - \frac{1}{- \frac{33}{94} + \frac{5 \sqrt{1149}}{94}}
\left(\frac{- \frac{3276}{- \sqrt{1149} - 22} + 276}{\left(- \sqrt{1149} -
22\right) \left(- \frac{59 \sqrt{1149}}{95} - \frac{1837}{95}\right)} -
\frac{13}{- \sqrt{1149} - 22}\right) \left(- \frac{26}{-22 + \sqrt{1149}} -
\frac{- \frac{273}{-22 + \sqrt{1149}} + 23}{- \frac{1837}{95} + \frac{59
\sqrt{1149}}{95}} + \frac{- \frac{6552}{-22 + \sqrt{1149}} + 552}{\left(-22 +
\sqrt{1149}\right) \left(- \frac{1837}{95} + \frac{59
\sqrt{1149}}{95}\right)}\right)\right) + \frac{1}{- \frac{33}{94} + \frac{5
\sqrt{1149}}{94}} \left(\frac{- \frac{13104}{- \sqrt{1149} - 22} + 1104}{\left(-
\sqrt{1149} - 22\right) \left(- \frac{59 \sqrt{1149}}{95} -
\frac{1837}{95}\right)} - \frac{52}{- \sqrt{1149} - 22}\right) - \frac{e^{33 +
\sqrt{1149}}}{- \frac{59 \sqrt{1149}}{95} - \frac{1837}{95}} \left(-
\frac{1}{\left(- \frac{33}{94} + \frac{5 \sqrt{1149}}{94}\right) \left(\frac{33
\sqrt{1149}}{2303} + \frac{5745}{2303}\right)} \left(-1 - \frac{1}{-
\frac{33}{94} + \frac{5 \sqrt{1149}}{94}} \left(\frac{26}{- \sqrt{1149} - 22} -
\frac{- \frac{6552}{- \sqrt{1149} - 22} + 552}{\left(- \sqrt{1149} - 22\right)
\left(- \frac{59 \sqrt{1149}}{95} - \frac{1837}{95}\right)} + 2\right)\right)
\left(- \frac{26}{-22 + \sqrt{1149}} - \frac{- \frac{273}{-22 + \sqrt{1149}} +
23}{- \frac{1837}{95} + \frac{59 \sqrt{1149}}{95}} + \frac{- \frac{6552}{-22 +
\sqrt{1149}} + 552}{\left(-22 + \sqrt{1149}\right) \left(- \frac{1837}{95} +
\frac{59 \sqrt{1149}}{95}\right)}\right) + \frac{2}{- \frac{33}{94} + \frac{5
\sqrt{1149}}{94}}\right) \left(- \frac{273}{- \sqrt{1149} - 22} + 23\right) &
\frac{\left(- \frac{273}{-22 + \sqrt{1149}} + 23\right) \left(\frac{13}{-
\sqrt{1149} - 22} - \frac{- \frac{3276}{- \sqrt{1149} - 22} + 276}{\left(-
\sqrt{1149} - 22\right) \left(- \frac{59 \sqrt{1149}}{95} -
\frac{1837}{95}\right)} + 1\right)}{\left(- \frac{1837}{95} + \frac{59
\sqrt{1149}}{95}\right) \left(- \frac{33}{94} + \frac{5 \sqrt{1149}}{94}\right)
\left(\frac{33 \sqrt{1149}}{2303} + \frac{5745}{2303}\right) e^{-33 +
\sqrt{1149}}} - \frac{2}{\left(- \frac{33}{94} + \frac{5 \sqrt{1149}}{94}\right)
\left(\frac{33 \sqrt{1149}}{2303} + \frac{5745}{2303}\right)} \left(-
\frac{13}{-22 + \sqrt{1149}} + \frac{- \frac{3276}{-22 + \sqrt{1149}} +
276}{\left(-22 + \sqrt{1149}\right) \left(- \frac{1837}{95} + \frac{59
\sqrt{1149}}{95}\right)} - \frac{1}{- \frac{33}{94} + \frac{5 \sqrt{1149}}{94}}
\left(\frac{- \frac{3276}{- \sqrt{1149} - 22} + 276}{\left(- \sqrt{1149} -
22\right) \left(- \frac{59 \sqrt{1149}}{95} - \frac{1837}{95}\right)} -
\frac{13}{- \sqrt{1149} - 22}\right) \left(- \frac{26}{-22 + \sqrt{1149}} -
\frac{- \frac{273}{-22 + \sqrt{1149}} + 23}{- \frac{1837}{95} + \frac{59
\sqrt{1149}}{95}} + \frac{- \frac{6552}{-22 + \sqrt{1149}} + 552}{\left(-22 +
\sqrt{1149}\right) \left(- \frac{1837}{95} + \frac{59
\sqrt{1149}}{95}\right)}\right)\right) \left(\frac{13}{- \sqrt{1149} - 22} -
\frac{- \frac{3276}{- \sqrt{1149} - 22} + 276}{\left(- \sqrt{1149} - 22\right)
\left(- \frac{59 \sqrt{1149}}{95} - \frac{1837}{95}\right)} + 1\right) +
\frac{1}{- \frac{33}{94} + \frac{5 \sqrt{1149}}{94}} \left(\frac{- \frac{6552}{-
\sqrt{1149} - 22} + 552}{\left(- \sqrt{1149} - 22\right) \left(- \frac{59
\sqrt{1149}}{95} - \frac{1837}{95}\right)} - \frac{26}{- \sqrt{1149} -
22}\right) - \frac{e^{33 + \sqrt{1149}}}{- \frac{59 \sqrt{1149}}{95} -
\frac{1837}{95}} \left(\frac{1}{\left(- \frac{33}{94} + \frac{5
\sqrt{1149}}{94}\right)^{2} \left(\frac{33 \sqrt{1149}}{2303} +
\frac{5745}{2303}\right)} \left(- \frac{26}{-22 + \sqrt{1149}} - \frac{-
\frac{273}{-22 + \sqrt{1149}} + 23}{- \frac{1837}{95} + \frac{59
\sqrt{1149}}{95}} + \frac{- \frac{6552}{-22 + \sqrt{1149}} + 552}{\left(-22 +
\sqrt{1149}\right) \left(- \frac{1837}{95} + \frac{59
\sqrt{1149}}{95}\right)}\right) \left(\frac{13}{- \sqrt{1149} - 22} - \frac{-
\frac{3276}{- \sqrt{1149} - 22} + 276}{\left(- \sqrt{1149} - 22\right) \left(-
\frac{59 \sqrt{1149}}{95} - \frac{1837}{95}\right)} + 1\right) + \frac{1}{-
\frac{33}{94} + \frac{5 \sqrt{1149}}{94}}\right) \left(- \frac{273}{-
\sqrt{1149} - 22} + 23\right) & \frac{1}{\frac{33 \sqrt{1149}}{2303} +
\frac{5745}{2303}} \left(- \frac{26}{-22 + \sqrt{1149}} + \frac{-
\frac{6552}{-22 + \sqrt{1149}} + 552}{\left(-22 + \sqrt{1149}\right) \left(-
\frac{1837}{95} + \frac{59 \sqrt{1149}}{95}\right)} - \frac{2}{- \frac{33}{94} +
\frac{5 \sqrt{1149}}{94}} \left(\frac{- \frac{3276}{- \sqrt{1149} - 22} +
276}{\left(- \sqrt{1149} - 22\right) \left(- \frac{59 \sqrt{1149}}{95} -
\frac{1837}{95}\right)} - \frac{13}{- \sqrt{1149} - 22}\right) \left(-
\frac{26}{-22 + \sqrt{1149}} - \frac{- \frac{273}{-22 + \sqrt{1149}} + 23}{-
\frac{1837}{95} + \frac{59 \sqrt{1149}}{95}} + \frac{- \frac{6552}{-22 +
\sqrt{1149}} + 552}{\left(-22 + \sqrt{1149}\right) \left(- \frac{1837}{95} +
\frac{59 \sqrt{1149}}{95}\right)}\right)\right) - \frac{- \frac{273}{-22 +
\sqrt{1149}} + 23}{\left(- \frac{1837}{95} + \frac{59 \sqrt{1149}}{95}\right)
\left(\frac{33 \sqrt{1149}}{2303} + \frac{5745}{2303}\right) e^{-33 +
\sqrt{1149}}} + \frac{e^{33 + \sqrt{1149}}}{\left(- \frac{33}{94} + \frac{5
\sqrt{1149}}{94}\right) \left(- \frac{59 \sqrt{1149}}{95} -
\frac{1837}{95}\right) \left(\frac{33 \sqrt{1149}}{2303} +
\frac{5745}{2303}\right)} \left(- \frac{273}{- \sqrt{1149} - 22} + 23\right)
\left(- \frac{26}{-22 + \sqrt{1149}} - \frac{- \frac{273}{-22 + \sqrt{1149}} +
23}{- \frac{1837}{95} + \frac{59 \sqrt{1149}}{95}} + \frac{- \frac{6552}{-22 +
\sqrt{1149}} + 552}{\left(-22 + \sqrt{1149}\right) \left(- \frac{1837}{95} +
\frac{59 \sqrt{1149}}{95}\right)}\right)\\- \frac{1}{- \frac{33}{94} + \frac{5
\sqrt{1149}}{94}} \left(\frac{- \frac{6552}{- \sqrt{1149} - 22} + 552}{\left(-
\sqrt{1149} - 22\right) \left(- \frac{59 \sqrt{1149}}{95} -
\frac{1837}{95}\right)} - \frac{26}{- \sqrt{1149} - 22}\right) -
\frac{1}{\frac{33 \sqrt{1149}}{2303} + \frac{5745}{2303}} \left(-1 - \frac{1}{-
\frac{33}{94} + \frac{5 \sqrt{1149}}{94}} \left(\frac{26}{- \sqrt{1149} - 22} -
\frac{- \frac{6552}{- \sqrt{1149} - 22} + 552}{\left(- \sqrt{1149} - 22\right)
\left(- \frac{59 \sqrt{1149}}{95} - \frac{1837}{95}\right)} + 2\right)\right)
\left(- \frac{13}{-22 + \sqrt{1149}} + \frac{- \frac{3276}{-22 + \sqrt{1149}} +
276}{\left(-22 + \sqrt{1149}\right) \left(- \frac{1837}{95} + \frac{59
\sqrt{1149}}{95}\right)} - \frac{1}{- \frac{33}{94} + \frac{5 \sqrt{1149}}{94}}
\left(\frac{- \frac{3276}{- \sqrt{1149} - 22} + 276}{\left(- \sqrt{1149} -
22\right) \left(- \frac{59 \sqrt{1149}}{95} - \frac{1837}{95}\right)} -
\frac{13}{- \sqrt{1149} - 22}\right) \left(- \frac{26}{-22 + \sqrt{1149}} -
\frac{- \frac{273}{-22 + \sqrt{1149}} + 23}{- \frac{1837}{95} + \frac{59
\sqrt{1149}}{95}} + \frac{- \frac{6552}{-22 + \sqrt{1149}} + 552}{\left(-22 +
\sqrt{1149}\right) \left(- \frac{1837}{95} + \frac{59
\sqrt{1149}}{95}\right)}\right)\right) + \frac{1}{\left(\frac{33
\sqrt{1149}}{2303} + \frac{5745}{2303}\right) e^{-33 + \sqrt{1149}}} \left(-1 -
\frac{1}{- \frac{33}{94} + \frac{5 \sqrt{1149}}{94}} \left(\frac{26}{-
\sqrt{1149} - 22} - \frac{- \frac{6552}{- \sqrt{1149} - 22} + 552}{\left(-
\sqrt{1149} - 22\right) \left(- \frac{59 \sqrt{1149}}{95} -
\frac{1837}{95}\right)} + 2\right)\right) + 1 + \left(- \frac{1}{\left(-
\frac{33}{94} + \frac{5 \sqrt{1149}}{94}\right) \left(\frac{33
\sqrt{1149}}{2303} + \frac{5745}{2303}\right)} \left(-1 - \frac{1}{-
\frac{33}{94} + \frac{5 \sqrt{1149}}{94}} \left(\frac{26}{- \sqrt{1149} - 22} -
\frac{- \frac{6552}{- \sqrt{1149} - 22} + 552}{\left(- \sqrt{1149} - 22\right)
\left(- \frac{59 \sqrt{1149}}{95} - \frac{1837}{95}\right)} + 2\right)\right)
\left(- \frac{26}{-22 + \sqrt{1149}} - \frac{- \frac{273}{-22 + \sqrt{1149}} +
23}{- \frac{1837}{95} + \frac{59 \sqrt{1149}}{95}} + \frac{- \frac{6552}{-22 +
\sqrt{1149}} + 552}{\left(-22 + \sqrt{1149}\right) \left(- \frac{1837}{95} +
\frac{59 \sqrt{1149}}{95}\right)}\right) + \frac{2}{- \frac{33}{94} + \frac{5
\sqrt{1149}}{94}}\right) e^{33 + \sqrt{1149}} & - \frac{1}{- \frac{33}{94} +
\frac{5 \sqrt{1149}}{94}} \left(\frac{- \frac{3276}{- \sqrt{1149} - 22} +
276}{\left(- \sqrt{1149} - 22\right) \left(- \frac{59 \sqrt{1149}}{95} -
\frac{1837}{95}\right)} - \frac{13}{- \sqrt{1149} - 22}\right) +
\frac{1}{\left(- \frac{33}{94} + \frac{5 \sqrt{1149}}{94}\right) \left(\frac{33
\sqrt{1149}}{2303} + \frac{5745}{2303}\right)} \left(- \frac{13}{-22 +
\sqrt{1149}} + \frac{- \frac{3276}{-22 + \sqrt{1149}} + 276}{\left(-22 +
\sqrt{1149}\right) \left(- \frac{1837}{95} + \frac{59 \sqrt{1149}}{95}\right)} -
\frac{1}{- \frac{33}{94} + \frac{5 \sqrt{1149}}{94}} \left(\frac{- \frac{3276}{-
\sqrt{1149} - 22} + 276}{\left(- \sqrt{1149} - 22\right) \left(- \frac{59
\sqrt{1149}}{95} - \frac{1837}{95}\right)} - \frac{13}{- \sqrt{1149} -
22}\right) \left(- \frac{26}{-22 + \sqrt{1149}} - \frac{- \frac{273}{-22 +
\sqrt{1149}} + 23}{- \frac{1837}{95} + \frac{59 \sqrt{1149}}{95}} + \frac{-
\frac{6552}{-22 + \sqrt{1149}} + 552}{\left(-22 + \sqrt{1149}\right) \left(-
\frac{1837}{95} + \frac{59 \sqrt{1149}}{95}\right)}\right)\right)
\left(\frac{13}{- \sqrt{1149} - 22} - \frac{- \frac{3276}{- \sqrt{1149} - 22} +
276}{\left(- \sqrt{1149} - 22\right) \left(- \frac{59 \sqrt{1149}}{95} -
\frac{1837}{95}\right)} + 1\right) - \frac{\frac{13}{- \sqrt{1149} - 22} -
\frac{- \frac{3276}{- \sqrt{1149} - 22} + 276}{\left(- \sqrt{1149} - 22\right)
\left(- \frac{59 \sqrt{1149}}{95} - \frac{1837}{95}\right)} + 1}{\left(-
\frac{33}{94} + \frac{5 \sqrt{1149}}{94}\right) \left(\frac{33
\sqrt{1149}}{2303} + \frac{5745}{2303}\right) e^{-33 + \sqrt{1149}}} +
\left(\frac{1}{\left(- \frac{33}{94} + \frac{5 \sqrt{1149}}{94}\right)^{2}
\left(\frac{33 \sqrt{1149}}{2303} + \frac{5745}{2303}\right)} \left(-
\frac{26}{-22 + \sqrt{1149}} - \frac{- \frac{273}{-22 + \sqrt{1149}} + 23}{-
\frac{1837}{95} + \frac{59 \sqrt{1149}}{95}} + \frac{- \frac{6552}{-22 +
\sqrt{1149}} + 552}{\left(-22 + \sqrt{1149}\right) \left(- \frac{1837}{95} +
\frac{59 \sqrt{1149}}{95}\right)}\right) \left(\frac{13}{- \sqrt{1149} - 22} -
\frac{- \frac{3276}{- \sqrt{1149} - 22} + 276}{\left(- \sqrt{1149} - 22\right)
\left(- \frac{59 \sqrt{1149}}{95} - \frac{1837}{95}\right)} + 1\right) +
\frac{1}{- \frac{33}{94} + \frac{5 \sqrt{1149}}{94}}\right) e^{33 + \sqrt{1149}}
& \frac{1}{\left(\frac{33 \sqrt{1149}}{2303} + \frac{5745}{2303}\right) e^{-33 +
\sqrt{1149}}} - \frac{1}{\frac{33 \sqrt{1149}}{2303} + \frac{5745}{2303}}
\left(- \frac{13}{-22 + \sqrt{1149}} + \frac{- \frac{3276}{-22 + \sqrt{1149}} +
276}{\left(-22 + \sqrt{1149}\right) \left(- \frac{1837}{95} + \frac{59
\sqrt{1149}}{95}\right)} - \frac{1}{- \frac{33}{94} + \frac{5 \sqrt{1149}}{94}}
\left(\frac{- \frac{3276}{- \sqrt{1149} - 22} + 276}{\left(- \sqrt{1149} -
22\right) \left(- \frac{59 \sqrt{1149}}{95} - \frac{1837}{95}\right)} -
\frac{13}{- \sqrt{1149} - 22}\right) \left(- \frac{26}{-22 + \sqrt{1149}} -
\frac{- \frac{273}{-22 + \sqrt{1149}} + 23}{- \frac{1837}{95} + \frac{59
\sqrt{1149}}{95}} + \frac{- \frac{6552}{-22 + \sqrt{1149}} + 552}{\left(-22 +
\sqrt{1149}\right) \left(- \frac{1837}{95} + \frac{59
\sqrt{1149}}{95}\right)}\right)\right) - \frac{e^{33 + \sqrt{1149}}}{\left(-
\frac{33}{94} + \frac{5 \sqrt{1149}}{94}\right) \left(\frac{33
\sqrt{1149}}{2303} + \frac{5745}{2303}\right)} \left(- \frac{26}{-22 +
\sqrt{1149}} - \frac{- \frac{273}{-22 + \sqrt{1149}} + 23}{- \frac{1837}{95} +
\frac{59 \sqrt{1149}}{95}} + \frac{- \frac{6552}{-22 + \sqrt{1149}} +
552}{\left(-22 + \sqrt{1149}\right) \left(- \frac{1837}{95} + \frac{59
\sqrt{1149}}{95}\right)}\right)\end{matrix}\right]
}
$$
Simplifying this result doesn't seem to help either:

\begin{minted}[]{python}
simplify(expr)
\end{minted}

$$
{
\scriptsize
\left[\begin{matrix}\frac{1}{12 \left(-1065889 + 33298 \sqrt{1149}\right)
e^{\sqrt{1149}}} \left(- 8625947 e^{33 + 2 \sqrt{1149}} - 2131778
e^{\sqrt{1149}} - 2032943 e^{33} + 74651 \sqrt{1149} e^{33} + 66596 \sqrt{1149}
e^{\sqrt{1149}} + 258329 \sqrt{1149} e^{33 + 2 \sqrt{1149}}\right) & \frac{1}{6
\left(-1065889 + 33298 \sqrt{1149}\right) e^{\sqrt{1149}}} \left(- 965995
e^{33 + 2 \sqrt{1149}} - 66596 \sqrt{1149} e^{\sqrt{1149}} - 1165783 e^{33} +
36081 \sqrt{1149} e^{33} + 2131778 e^{\sqrt{1149}} + 30515 \sqrt{1149} e^{33 + 2
\sqrt{1149}}\right) & \frac{1}{6 \left(-43187463 + 1274291 \sqrt{1149}\right)
e^{\sqrt{1149}}} \left(- 2723224 \sqrt{1149} e^{33 + 2 \sqrt{1149}} - 43187463
e^{\sqrt{1149}} - 49129419 e^{33} + 1448933 \sqrt{1149} e^{33} + 1274291
\sqrt{1149} e^{\sqrt{1149}} + 92316882 e^{33 + 2 \sqrt{1149}}\right)\\\frac{1}{6
\left(-1065889 + 33298 \sqrt{1149}\right) e^{\sqrt{1149}}} \left(- 66949 e^{33 +
2 \sqrt{1149}} - 66596 \sqrt{1149} e^{\sqrt{1149}} - 2064829 e^{33} + 61128
\sqrt{1149} e^{33} + 2131778 e^{\sqrt{1149}} + 5468 \sqrt{1149} e^{33 + 2
\sqrt{1149}}\right) & \frac{1}{3 \left(4213 + 125 \sqrt{1149}\right)
e^{\sqrt{1149}}} \left(44 e^{33} + 2 \sqrt{1149} e^{33} + 8426 e^{\sqrt{1149}} +
250 \sqrt{1149} e^{\sqrt{1149}} + 4169 e^{33 + 2 \sqrt{1149}} + 123 \sqrt{1149}
e^{33 + 2 \sqrt{1149}}\right) & \frac{1}{6 \left(-43187463 + 1274291
\sqrt{1149}\right) e^{\sqrt{1149}}} \left(- 78841 \sqrt{1149} e^{33 + 2
\sqrt{1149}} - 2548582 \sqrt{1149} e^{\sqrt{1149}} - 89061939 e^{33} + 2627423
\sqrt{1149} e^{33} + 86374926 e^{\sqrt{1149}} + 2687013 e^{33 + 2
\sqrt{1149}}\right)\\\frac{1}{12 \left(-1065889 + 33298 \sqrt{1149}\right)
e^{\sqrt{1149}}} \left(- 236457 \sqrt{1149} e^{33 + 2 \sqrt{1149}} - 6226373
e^{33} - 2131778 e^{\sqrt{1149}} + 66596 \sqrt{1149} e^{\sqrt{1149}} + 169861
\sqrt{1149} e^{33} + 8358151 e^{33 + 2 \sqrt{1149}}\right) & \frac{1}{6
\left(-1065889 + 33298 \sqrt{1149}\right) e^{\sqrt{1149}}} \left(- 25145
\sqrt{1149} e^{33 + 2 \sqrt{1149}} - 66596 \sqrt{1149} e^{\sqrt{1149}} - 3163663
e^{33} + 91741 \sqrt{1149} e^{33} + 2131778 e^{\sqrt{1149}} + 1031885 e^{33 + 2
\sqrt{1149}}\right) & \frac{1}{6 \left(-43187463 + 1274291 \sqrt{1149}\right)
e^{\sqrt{1149}}} \left(- 86942856 e^{33 + 2 \sqrt{1149}} - 128994459 e^{33} -
43187463 e^{\sqrt{1149}} + 1274291 \sqrt{1149} e^{\sqrt{1149}} + 3805913
\sqrt{1149} e^{33} + 2565542 \sqrt{1149} e^{33 + 2
\sqrt{1149}}\right)\end{matrix}\right]
}
$$

Methods suggested online only provide numerical solutions or partial sums:

\begin{itemize}
\item \href{https://stackoverflow.com/questions/47240208/sympy-symbolic-matrix-exponential}{python -
Sympy Symbolic Matrix Exponential - Stack Overflow}
\item \href{https://stackoverflow.com/a/50718831/12843551}{python - Exponentiate
symbolic matrix expression using SymPy - Stack Overflow}
\item \href{https://stackoverflow.com/a/54025116/12843551}{Calculate state transition
matrix in python - Stack Overflow}
\end{itemize}

Instead this will need to be implemented from first principles.

\subsection{Theory}
\label{sec:org883d33f}
\subsubsection{Matrix Exponentiation}
\label{sec:orgf5820a2}
A Matrix Exponential is defined by using the ordinary exponential power series \cite[Ch. 2]{hallLieGroupsLie2015},\cite[Ch. 8.4]{Zil2009} (should we prove the power series generally?):

\begin{align}
    e^{\mathit{\mathbf{X}}} = \sum^{\infty}_{k= 0}   \left[ \frac{1}{k!} \cdot  \mathit{\mathbf{X^k}} \right] 
\end{align}

This definition can be expanded upon however by using properties of logarithms:

\begin{align}
    b &= e^{\log_e{\left( b \right) }}, \quad \forall b \in \mathbb{C} \label{eq:bydef}\\
 \implies  b^{\mathbf{X}}&= \left( e^{\log_e{\left( b \right) }} \right)^{\mathbf{X}} \label{eq:tojustify} \\
  \implies  b^{\mathbf{X}} &= e^{\log_e{b}  \mathbf{X} }
\end{align}

The identity in \eqref{eq:bydef} is justified by the definition of the complex log. However some discussion is required for \eqref{eq:tojustify}  because it is not clear that the
exponential will generally distribute throught he parenthesis like so \(\left( a\cdot b \right)^{k} = a^k\cdot b^k\), for example
consider \(\left( \left[ - 1 \right]^2 \cdot 3
\right)^{\frac{1}{2}} \neq \left[ - 1 \right]^{\frac{2}{2}} \cdot
3^{\frac{1}{2}}\).

A sufficient condition for this identity is \(k \in
\mathbb{Z}^{*}\), consider this example which will be important later:

\begin{align}
    \left( \log_e{\left( b \right) }\mathbf{X} \right)^{k} , \quad \forall k \in \mathbb{Z^{*}}
\end{align}

Because multiplication is commutative \(\forall z \in \mathbb{C}\), this could be
re-expressed in the form:

\begin{align}
 \left( \log_e{\left( b \right) }\mathbf{X} \right)^{k} &=    \underbrace{\log_e{\left( b \right) }\cdot  \log_e{\left( b \right) } \cdot  \log_e{\left( b \right) }\ldots }_{k \text{ times}} \times \underbrace{\mathbf{X}\mathbf{X}\mathbf{X}\ldots}_{k \text{ times}} \notag \\
 &= \log_e^k{\left( b \right) } \mathbf{X}^k \label{eq:matpower}
\end{align}

Now consider the the following by applying \eqref{eq:matpower}:

\begin{align}
    e^{X}&= \sum^{\infty}_{k= 0}   \left[ \frac{1}{k!} \mathbf{X}^{k} \right]  \notag \\
    \implies  e^{bX}&= \sum^{\infty}_{k= 0}   \left[ \frac{1}{k!} \left( b\mathbf{X} \right)^{k} \right] \quad \forall b \in \mathbb{C} \notag \\
    &= \sum^{\infty}_{k= 0}   \left[ \frac{1}{k!}b^k \mathbf{X}^k \right] \notag \\
    &= \left( e^b \right)^{\mathbf{X}} \notag \\
    &\implies  e^{b \mathbf{X}} = e^{\mathbf{X}b}= \left( e^b \right)^{\mathbf{X}}= \left( e^{\mathbf{X}} \right)^b  \qquad \qquad \square \label{eq:expmatpower}
\end{align}

So the matrix exponential for an arbitrary base could be given by:

\begin{align}
   b = e^{\log_e{\left( b \right) }}, \quad \forall b \in \mathbb{C} \notag \\
    \implies  b^{\mathbf{X}} &= \left( e^{\log_e{\left( b \right) }} \right)^{\mathbf{X}} \notag \\
     \text{as per \eqref{eq:expmatpower}} \notag \\
    b^{\mathbf{X}} &=  e^{\log_e{\left( b \right) } {\mathbf{X}}}  \notag \\
     b^{\mathbf{X}} &= \sum^{\infty}_{k= 0}   \left[ \frac{\left( \log_e{\left( b \right) }\mathbf{X} \right)^k}{k!} \right]  \notag \\
     &= \sum^{\infty}_{k= 0}   \left[ \frac{\log_e ^{k}{\left( b \right) }}{k!}\mathbf{X}^{k} \right]
\end{align}

This is also consistent with the \emph{McLaurin Series} expansion of \(b^{\mathbf{X}}
\enspace (\forall b \in \mathbb{C})\):

\begin{align*}
f\left( x \right) &= \sum^{\infty}_{k= 0}   \left[ \frac{f^{\left( n \right)}\left( 0 \right)}{k!} x^{k} \right]  \\
\implies  b^x &= \sum^{\infty}_{k= 0}  \left[ \frac{\frac{\mathrm{d}^n }{\mathrm{d} x^n}\left( b^x \right) \vert_{x=0}   }{k!} x^k \right]  \\
\implies  b^{\mathbf{X}} &= \sum^{\infty}_{k= 0}   \left[ \frac{\frac{\mathrm{d}^n }{\mathrm{d}\mathbf{X}^n  } \left( b^{\mathbf{X}} \right) \vert_{\mathbf{X}= \mathbf{O}}}{k!} \mathbf{X}^k \right]
\end{align*}

By ordinary calculus identities we have\(f\left( x \right) = b^{x} \implies
f^{\left( n \right)}\left( x \right) = b^{x} \log_e^n{\left( b \right)}\) which
distribute through a matrix and hence:

\begin{align*}
    b^x &= \sum^{\infty}_{k= 0}  \left[ \frac{b^0 \log_e^k{\left( b \right) }}{k!} x^k \right]  \\
    \implies  b^{\mathbf{X}} &= \sum^{\infty}_{k= 0}  \left[ \frac{b^0 \log_e^k{\left( b \right) }}{k!} \mathbf{X}^k \right]
\end{align*}

By the previous identity:

\begin{align*}
\implies  b^{\mathbf{X}} &= \sum^{\infty}_{k= 0}  {\left[ \frac{{\left( \log_e{\left( b \right) } \mathbf{X} \right)}^k}{k!} \right]} \\
    &= e^{\log_e{\left( b \right) } \mathbf{X}}
\end{align*}

\subsubsection{Matrix-Matrix Exponentiation}
\label{sec:orgf512705}

Matrix-Matrix exponentiation has applications in quantum mechanics \cite[p. 84]{barradasIteratedExponentiationMatrixMatrix1994}.

As for Matrices with the requirements:

\begin{enumerate}
\item Square
\item Normal:
\begin{itemize}
\item Commutes with it's congugate transpose
\end{itemize}
\item Non Singular
\item Non Zero Determinant
\end{enumerate}

\begin{align*}
    \left| \left| A-I \right| \right|<1  &\implies  e^{\log_e{\left( \mathbf{A} \right) }} = \mathbf{A} \enspace \text{(By Lie Groups Springer Textbook)}\\
                     &\implies  \mathbf{A}^{\mathbf{B}} =\left( e^{\log_e{\left( \mathbf{A} \right) }} \right)^{\mathbf{B}} \\
             & \text{Similar justification as \eqref{eq:expmatpower}} \\
             & \implies  \mathbf{A}^{\mathbf{B}}= e^{\log_e{\left( \mathbf{A} \right) } \mathbf{B}}
\end{align*}

However the following identities are by \textbf{Definition} anyway: \cite{barradasIteratedExponentiationMatrixMatrix1994}

\begin{align}
\mathbf{A}^{\mathbf{B}}&= e^{\log_e{\left( \mathbf{A} \right) } \mathbf{B}} \\
\ ^{\mathbf{B}} \mathbf{A} &= e^{ \mathbf{B} \log_e{\left( \mathbf{A} \right) } }
\end{align}

\subsection{An alternative Implementation in Sympy}
\label{sec:org3723b27}

\begin{minted}[]{python}
def matexp(mat, base = E):
      """
      Return the Matrix Exponential of a square matrix
      """
      import copy
      import sympy
  # Should realy test for sympy vs numpy array
  # Test for Square Matrix
      if mat.shape[0] != mat.shape[1]:
          print("ERROR: Only defined for Square matrices")
          return
      m = zeros(mat.shape[0])
      for i in range(m.shape[0]):
          for j in range(m.shape[1]):
              m[i,j] = Sum((mat[i,j]*ln(base))**k/factorial(k), (k, 0, oo)).doit()
      return m
\end{minted}

\begin{minted}[]{python}
matexp(A, pi)
\end{minted}

$$
\left[\begin{matrix}\pi^{11} & \pi^{12} & \pi^{13}\\\pi^{21} & \pi^{22} &
\pi^{23}\\\pi^{31} & \pi^{32} & \pi^{33}\end{matrix}\right]
$$

But it would be nice to expand this to matrix bases for there uses in quantum
mechanics.

The built in method for a**mat is not implemented.

there is exp(mat) but this returns garbage (see github issue), (see other
solution on stack exchange that is numeric and example)

show our method with proofs of

cauchy power taylor then exp

then show our code

\begin{minted}[]{python}
A = Matrix([ [11,12,13], [21,22,23], [31,32,33] ])

  B = Matrix([
      [1,2,3],
      [4,5,6],
      [7,8,9]
  ])


  A**B
\end{minted}
\section[Recursive Relations]{Recursive Relations\hfill{}\textsc{Ryan}}
\label{sec:org187decc}
A linear recurrence relation is of the form:

\begin{align}
\sum^{\infty}_{n= 0}   \left[ c_i \cdot  a_n \right] = 0, \quad \exists c \in
\mathbb{R}, \enspace \forall i<k\in\mathbb{Z}^+ \label{eq:recurrence-relation-def}
\end{align}

In order to find a solution for \(a_n\), the following one-to-one
correspondence can be used to relate the vector space of the sequence to the
power series ring:(cite stackExchange[1]):

\begin{align}
g: \left( a_n \right)_{n\in\mathbb{Z}^+} \rightarrow \mathbb{C}\left[ \left[ x \right]  \right]: g\left( a_n \right) = \sum^{\infty}_{n= 0}\left[ \frac{x^n}{n!} a_n \right] \label{eq:gen-func-def}
\end{align}

This technique is referred to as generating functions.
\cite{lehmanReadingsMathematicsComputer2010}


\subsection{Generating Functions}
\label{sec:org739ccfe}
\subsubsection{Generating Functions}
\label{sec:org200f07c}
A \href{https://en.wikipedia.org/wiki/Generating\_function}{Generating Function}, as shown in \eqref{eq:gen-func-def}  is a way of encoding an \href{https://en.wikipedia.org/wiki/Infinite\_sequence}{infinite series} of numbers (\(a_n\))
by treating them as the coefficients of a power series (\(\sum^\infty_{n = 0}
\left[ a_nx^n \right]\)) where the variable remains in an indeterminate form,
they were first introduced by Abraham De Moivre in 1730 in order to solve the
linear recurrence relations \cite{knuthArtComputerProgramming1997}, as shown in \eqref{eq:recurrence-relation-def} (such as the \emph{Fibonacci Sequence}).

\subsubsection{Example}
\label{sec:orge4b7317}
Given the Linear Recurrence Relation:

\begin{align*}
a_0= 1 \\
a_0= 1 \\
a_{n+  2} =  a_{n+  1 +  2 a_n}, \quad n \geq 0
\end{align*}

To solve this we can use what's known as a
\href{https://en.wikipedia.org/wiki/Generating\_function}{Generating
Function}, \hyperref[sec:org200f07c]{see the disucssion below}

We will make consider the function \(f(x)\) as shown below in:

\begin{align}
f\left( x \right)= \sum^{\infty}_{n= 0}   \left[ a_nx^n \right] \label{eq:pow-gen-func-np0}
\end{align}


It can be shown (see \eqref{iterate-pow-gen-func}) that:


\begin{align}
    \sum^{\infty}_{n= 0}  \left[ a_{n+  1} x^n \right] &= \frac{f\left( x \right)- a_0}{x} \label{eq:pow-gen-func-np1} \\
\sum^{\infty}_{n= 0}  \left[ a_{n+  2} x^n \right]  &= \frac{f\left( x \right) - a_0 - a_1x}{x^2} \label{eq:pow-gen-func-np2}
\end{align}

So to use the generating Function consider:

\begin{align}
    2a_n +  a_{n+  1 }&= a_{n+  2} \nonumber \\
    2a_nx^n +  a_{n+  1 } x^n &= a_{n+  2} x^n \nonumber \\
    \sum^{\infty}_{n= 0}   \left[ 2a_nx^n \right] + \sum^{\infty}_{n= 0}   \left[  a_{n+  1 } x^n  \right]   &= \sum^{\infty}_{n= 0}   \left[ a_{n+  2} x^n   \right] \label{eq:series-rep-pow-example}
\end{align}

Observe that in \eqref{eq:series-rep-pow-exampluse} tuse te

By applying the previous identity shown in \eqref{eq:pow-gen-func-np0}, \eqref{eq:pow-gen-func-np1} and \eqref{eq:pow-gen-func-np2}:

\begin{align}
2f\left( x \right) +  \frac{f\left( x \right)- a_0}{x} &= \frac{f\left( x \right)- a_0}{- a_1x}x^2 \nonumber \\
\implies  f\left( x \right) &=  \frac{1}{1- x- x^2} \label{eq:power-series-form-example}
\end{align}

\begin{center}
\begin{tabular}{l}
WARNING\\
\hline
I accidently dropped the \(2\) here, it doesn't matter but it does show that how this could be dealt with algebraically\\
\end{tabular}
\end{center}

The function \(f(x)\) in \eqref{eq:power-series-form-example} can be solved by way of a power series, ( see for example \href{./University/Analysis/11\_Series.md}{11\textsubscript{Series}}), but first it is
necessary to use partial fractions to split it up:


By partial fractions it is known:

\begin{align*}
    f\left( x \right)&= \frac{1}{1- x- x^2}\\
&= \frac{- 1}{x^2 +  x -  1}\\
&= \frac{- 1}{\left( x- 2 \right)\left( x- 1 \right)}\\
&= \frac{A_1}{x- 2}+  \frac{A_2}{x- 1}, \quad A_i \in \mathbb{R}, i \in \mathbb{Z}^+ \\
 \implies  - 1 &= A_1\left( x- 1 \right) +  A_2\left( x- 2 \right)\\
 \text{Let $x$ = 2:}\\
 - 1&= A_1\left( 2-1 \right) +  0 \\
&= A_1 = - 1 \\
 \text{Let $x$ = 1:}\\
 - 1 &=  0 +  A_2 \left( 1- 2 \right) \\
 \implies  A_2&= 1 \\
 \text{Hence:}\\
 f\left( x \right)&=    \frac{1}{x- 1} - \frac{1}{x- 2}
\end{align*}

By definition of the power series:

\begin{align}
\sum^{\infty}_{n= 0}\left[ rx^n \right] = \frac{1}{1- rx^n} \label{eq:pow-series-definition}
\end{align}

we can conclude that:

\begin{align*}
\frac{1}{x- 1}&= -\frac{1}{1 -\left( 1 \right) x} \\
&= -\sum^{\infty}_{n= 0}\left[ x^n \right]  \\
\frac{-1}{x- 2} &= \frac{1}{2- x} \\
&= \frac{1}{2}\frac{1}{1-\frac{1}{2}x} \\
&= \frac{1}{2} \sum^{\infty}_{n= 0}\left[ \left( \frac{1}{2}x \right) ^n \right]
\end{align*}

and so:

\begin{align*}
f\left( x \right) &= \frac{1}{2}\sum^{\infty}_{n= 0}\left[ \left( \frac{1}{2}x \right) ^n \right] - \sum^{\infty}_{n= 0}\left[ x^n \right] \\
f\left( x \right) &= \sum^{\infty}_{n= 0}\left[ \frac{1}{2}\left( \frac{1}{2}x \right) ^n -x^n \right]  \\
f\left( x \right) &= \sum^{\infty}_{n= 0}\left[ \frac{1}{2 \cdot 2^n} x^n -x^n \right]  \\
f\left( x \right) &= \sum^{\infty}_{n= 0}\left[x^n {\left( {\frac{1}{2 \cdot 2^n} -1} \right) } \right]  \\
\end{align*}
\begin{align}
 \implies  a_n &= \frac{1}{2 \cdot 2^n} - 1 \label{eq:seq-end-value}
\end{align}

\subsection{Exponential Generating Function}
\label{sec:orgfcd5233}
\subsubsection{Motivation}
\label{sec:org44082bb}
Consider the \emph{Fibonacci Sequence}:


\begin{align}
    a_{n}&= a_{n - 1} + a_{n - 2} \nonumber \\
\iff a_{n+  2} &= a_{n+  1} +  a_n \label{eq:fib-def}
\end{align}


Solving this outright is quite difficult, a power
series generating function can be used to solve it as shown in section \ref{sec:orge4b7317}, which provides a corresponding equation to the effect of:


\begin{align*}
x^2 f\left( x \right) -  x f \left( x \right) -  f\left( x \right)=  0
\end{align*}

This however still requires a little intuition, however, just from observation, this appears similar in structure to
the following \emph{ordinary differential equation}, which would be fairly easy to deal with:


\begin{align*}
f''\left( x \right)- f'\left( x \right)- f\left( x \right)=  0
\end{align*}


This would imply that \(f\left( x \right) \propto e^{mx}, \quad \exists m \in \mathbb{Z}\) because
\(\frac{\mathrm{d}\left( e^x \right) }{\mathrm{d} x} = e^x\), but that's
fine because we have a power series for that already:


\begin{align*}
f\left( x \right)= e^{rx} = \sum^{\infty}_{n= 0}   \left[ r \frac{x^n}{n!} \right]
\end{align*}


So this would give an easy means by which to solve the linear recurrence
relation.

\subsubsection{Example}
\label{sec:orgd72c493}
Consider using the following generating function, (the derivative of the generating function as in \eqref{eq:exp-gen-def-2} and \eqref{eq:exp-gen-def-3} is provided in section \ref{sec:org92cf896})




\begin{alignat}{2}
    f \left( x \right) &=  \sum^{\infty}_{n= 0}   \left[ a_{n} \cdot  \frac{x^n}{n!} \right]   &= e^x \label{eq:exp-gen-def-1} \\
    f'\left( x \right) &=  \sum^{\infty}_{n= 0}   \left[ a_{n+1} \cdot  \frac{x^n}{n!} \right]  &= e^x  \label{eq:exp-gen-def-2} \\
    f''\left( x \right) &=  \sum^{\infty}_{n= 0}   \left[ a_{n+2} \cdot  \frac{x^n}{n!} \right] &= e^x  \label{eq:exp-gen-def-3}
\end{alignat}


So the recursive relation from \ref{eq:fib-def}  could be expressed :


\begin{align*}
a_{n+  2}    &= a_{n+  1} +  a_{n}\\
\frac{x^n}{n!}   a_{n+  2}    &= \frac{x^n}{n!}\left( a_{n+  1} +  a_{n}  \right)\\
\sum^{\infty}_{n= 0} \left[ \frac{x^n}{n!}   a_{n+  2} \right]        &= \sum^{\infty}_{n= 0}   \left[ \frac{x^n}{n!} a_{n+  1} \right]  + \sum^{\infty}_{n= 0}   \left[ \frac{x^n}{n!} a_{n}  \right]  \\
f''\left( x \right) &= f'\left( x \right)+  f\left( x \right)
\end{align*}


Using the theory of higher order linear differential equations with
constant coefficients it can be shown:


\begin{align*}
f\left( x \right)= c_1 \cdot  \mathrm{exp}\left[ \left( \frac{1- \sqrt{5} }{2} \right)x \right] +  c_2 \cdot  \mathrm{exp}\left[ \left( \frac{1 +  \sqrt{5} }{2} \right) \right]
\end{align*}


By equating this to the power series:


\begin{align*}
f\left( x \right)&= \sum^{\infty}_{n= 0}   \left[ \left( c_1\left( \frac{1- \sqrt{5} }{2} \right)^n +  c_2 \cdot  \left( \frac{1+ \sqrt{5} }{2} \right)^n \right) \cdot  \frac{x^n}{n} \right]
\end{align*}


Now given that:


\begin{align*}
f\left( x \right)= \sum^{\infty}_{n= 0}   \left[ a_n \frac{x^n}{n!} \right]
\end{align*}


We can conclude that:


\begin{align*}
a_n = c_1\cdot  \left( \frac{1- \sqrt{5} }{2} \right)^n +  c_2 \cdot  \left( \frac{1+  \sqrt{5} }{2} \right)
\end{align*}


By applying the initial conditions:


\begin{align*}
a_0= c_1 +  c_2  \implies  c_1= - c_2\\
a_1= c_1 \left( \frac{1+ \sqrt{5} }{2} \right) -  c_1 \frac{1-\sqrt{5} }{2}  \implies  c_1 = \frac{1}{\sqrt{5} }
\end{align*}


And so finally we have the solution to the \emph{Fibonacci Sequence} \ref{eq:fib-def}:


\begin{align}
    a_n &= \frac{1}{\sqrt{5} } \left[ \left( \frac{1+  \sqrt{5} }{2}  \right)^n -  \left( \frac{1- \sqrt{5} }{2} \right)^n \right] \nonumber \\
&= \frac{\varphi^n - \psi^n}{\sqrt{5} } \nonumber\\
&=\frac{\varphi^n -  \psi^n}{\varphi - \psi} \label{eq:fib-sol}
\end{align}


where:

\begin{itemize}
\item \(\varphi = \frac{1+ \sqrt{5} }{2} \approx 1.61\ldots\)
\item \(\psi = 1-\varphi = \frac{1- \sqrt{5} }{2} \approx 0.61\ldots\)
\end{itemize}

\subsubsection{Derivative of the Exponential Generating Function}
\label{sec:org92cf896}
Differentiating the exponential generating function has the effect of shifting the sequence to the backward: \cite{lehmanReadingsMathematicsComputer2010}

\begin{align}
    f\left( x \right) &= \sum^{\infty}_{n= 0}   \left[ a_n \frac{x^n}{n!} \right] \label{eq:exp-pow-series} \\
f'\left( x \right)) &= \frac{\mathrm{d} }{\mathrm{d} x}\left( \sum^{\infty}_{n= 0}   \left[ a_n \frac{x^n}{n!} \right]  \right) \nonumber \\
&= \frac{\mathrm{d}}{\mathrm{d} x} \left( a_0 \frac{x^0}{0!} +  a_1 \frac{x^1}{1!} +  a_2 \frac{x^2}{2!}+  a_3 \frac{x^3}{3! } +  \ldots \frac{x^k}{k!} \right) \nonumber \\
&= \sum^{\infty}_{n= 0}   \left[ \frac{\mathrm{d} }{\mathrm{d} x}\left( a_n \frac{x^n}{n!} \right) \right] \nonumber \\
&= \sum^{\infty}_{n= 0}   {\left[{ \frac{a_n}{{\left({ n- 1 }\right)!}} } x^{n- 1}  \right]} \nonumber \\
\implies f'(x) &= \sum^{\infty}_{n= 0}   {\left[{ \frac{x^n}{n!}a_{n+  1} }\right]} \label{eq:exp-pow-series-sol}
\end{align}

If \(f\left( x \right)= \sum^{\infty}_{n= 0 } \left[ a_n \frac{x^n}{n!} \right]\) can it be shown by induction that \(\frac{\mathrm{d}^k }{\mathrm{d} x^k} \left(  f\left( x \right) \right)= f^{k} \left( x \right) \sum^{\infty}_{n= 0}   \left[ x^n \frac{a_{n+  k}}{n!} \right]\)

\subsubsection{Homogeneous Proof}
\label{sec:orgf69b0f3}
An equation of the form:

\begin{align}
\sum^{\infty}_{n=0} \left[ c_{i} \cdot f^{(n)}(x) \right] = 0 \label{eq:hom-ode}
\end{align}

is said to be a homogenous linear ODE:

\begin{description}
\item[{Linear}] because the equation is linear with respect to \(f(x)\)
\item[{Ordinary}] because there are no partial derivatives (e.g. \(\frac{\partial }{\partial x}{\left({ f{\left({ x }\right)} }\right)}\)  )
\item[{Differential}] because the derivates of the function are concerned
\item[{Homogenous}] because the \textbf{\emph{RHS}} is 0
\begin{itemize}
\item A non-homogeous equation would have a non-zero RHS
\end{itemize}
\end{description}

There will be \(k\) solutions to a \(k^{\mathrm{th}}\) order linear ODE, each may be summed to produce a superposition which will also be a solution to the equation, \cite[Ch. 4]{Zil2009}  this will be considered as the desired complete solution (and this will be shown to be the only solution for the recurrence relation \eqref{eq:recurrence-relation-def}). These \(k\) solutions will be in one of two forms:

\begin{enumerate}
\item \(f(x)=c_{i} \cdot e^{m_{i}x}\)
\item \(f(x)=c_{i} \cdot x^{j}\cdot e^{m_{i}x}\)
\end{enumerate}

where:

\begin{itemize}
\item \(\sum^{k}_{i=0}\left[  c_{i}m^{k-i} \right] = 0\)
\begin{itemize}
\item This is referred to the characteristic equation of the recurrence relation or ODE \cite{levinSolvingRecurrenceRelations2018}
\end{itemize}
\item \(\exists i,j \in \mathbb{Z}^{+} \cap \left[0,k\right]\)
\begin{itemize}
\item These is often referred to as repeated roots \cite{levinSolvingRecurrenceRelations2018,zillMatrixExponential2009} with a multiplicity corresponding to the number of repetitions of that root \cite[\textsection 3.2]{nicodemiIntroductionAbstractAlgebra2007}
\end{itemize}
\end{itemize}

\paragraph{Unique Roots of Characteristic Equation}
\label{sec:org516b254}
\subparagraph{Example}
\label{sec:org2c8a7a1}
An example of a recurrence relation with all unique roots is the fibonacci sequence, as described in section \ref{sec:orgd72c493}.
\subparagraph{Proof}
\label{sec:org50e07b8}
Consider the linear recurrence relation \eqref{eq:recurrence-relation-def}:

\begin{align}
\sum^{\infty}_{n= 0}   \left[ c_i \cdot  a_n \right] = 0, \quad \exists c \in
\mathbb{R}, \enspace \forall i<k\in\mathbb{Z}^+ \tag{$\textrm{\ref{eq:recurrence-relation-def}}^2$}
\end{align}

By implementing the exponential generating function as shown in \eqref{eq:exp-gen-def-1}, this provides:


\begin{align}
    \sum^{k}_{i= 0}   {\left[{ c_i \cdot a_n } \right]} = 0 \nonumber \\
    \intertext{By Multiplying through and summing: } \notag \\
     \implies  \sum^{k}_{i= 0}   {\left[{ \sum^{\infty}_{n= 0}   {\left[{ c_i a_n \frac{x^n}{n!} }\right]}  }\right]}  \nonumber = 0 \\
     \sum^{k}_{i= 0}    {\left[{ c_i \sum^{\infty}_{n= 0}   {\left[{  a_n \frac{x^n}{n!} }\right]}  }\right]}  \nonumber = 0 \\
\end{align}

Recall from \eqref{eq:exp-gen-def-1} the generating function \(f{\left({ x }\right)}\):

\begin{align}
\sum^{k}_{i= 0}   {\left[{ c_i f^{{\left({ k }\right)} } } {\left({ x }\right)} \right]} \label{eq:exp-gen-def-proof}  &= 0
\end{align}


Now assume that the solution exists and all roots of the characteristic polynomial are unique (i.e. the solution is of the form \(f{\left({ x }\right)} \propto e^{m_i x}: \quad m_i \neq m_j \forall i\neq j\)), this implies that \cite[Ch. 4]{Zil2009} :

\begin{align}
    f{\left({ x }\right)} = \sum^{k}_{i= 0}   {\left[{ k_i e^{m_i x} }\right]}, \quad \exists m,k \in \mathbb{C} \nonumber
\end{align}

This can be re-expressed in terms of the exponential power series, in order to relate the solution of the function \(f{\left({ x }\right)}\) back to a solution of the sequence \(a_n\), (see section \ref{sec:orgcb7b0a9} for a derivation of the exponential power series):

\begin{align}
    \sum^{k}_{i= 0}   {\left[{ k_i e^{m_i x}  }\right]}  &= \sum^{k}_{i= 0}   {\left[{ k_i \sum^{\infty}_{n= 0}   \frac{{\left({ m_i x }\right)}^n}{n!}  }\right]}  \nonumber \\
							 &= \sum^{k}_{i= 0}  \sum^{\infty}_{n= 0}   k_i m_i^n \frac{x^n}{n!} \nonumber\\
							 &=    \sum^{\infty}_{n= 0} \sum^{k}_{i= 0}   k_i m_i^n \frac{x^n}{n!} \nonumber \\
							 &= \sum^{\infty}_{n= 0} {\left[{ \frac{x^n}{n!}  \sum^{k}_{i=0}   {\left[{ k_im^n_i }\right]}  }\right]}, \quad \exists k_i \in \mathbb{C}, \enspace \forall i \in \mathbb{Z}^+\cap {\left[{ 1, k }\right]}     \label{eq:unique-root-sol-power-series-form}
\end{align}

Recall the definition of the generating function from \ref{eq:exp-gen-def-proof}, by relating this to \eqref{eq:unique-root-sol-power-series-form}:

\begin{align}
    f{\left({ x }\right)} &= \sum^{\infty}_{n= 0}   {\left[{  \frac{x^n}{n!} a_n }\right]} \nonumber \\
&= \sum^{\infty}_{n= 0} {\left[{ \frac{x^n}{n!}  \sum^{k}_{i=0}   {\left[{ k_im^n_i }\right]}  }\right]}  \nonumber \\
      \implies  a_n &= \sum^{k}_{n= 0} {\left[{ k_im_i^n }\right]}     \nonumber \\ \nonumber
\square
\end{align}

This can be verified by the fibonacci sequence as shown in section \ref{sec:orgd72c493}, the solution to the characteristic equation is \(m_1 = \varphi, m_2 = {\left({ 1-\varphi }\right)}\) and the corresponding solution to the linear ODE and recursive relation are:

\begin{alignat}{4}
    f{\left({ x }\right)} &= &c_1 e^{\varphi x} +  &c_2 e^{{\left({ 1-\varphi }\right)} x}, \quad &\exists c_1, c_2 \in \mathbb{R} \subset \mathbb{C} \nonumber \\
    \iff  a_n &= &k_1 n^{\varphi} +  &k_2 n^{1- \varphi}, &\exists k_1, k_2 \in \mathbb{R} \subset \mathbb{C} \nonumber
\end{alignat}

\paragraph{Repeated Roots of Characteristic Equation}
\label{sec:org031646c}
\subparagraph{Example}
\label{sec:orgb43251d}
Consider the following recurrence relation:

\begin{align}
    a_n -  10a_{n+ 1} +  25a_{n+  2}&= 0 \label{eq:hom-repeated-roots-recurrence} \\
    \implies  \sum^{\infty}_{n= 0}   {\left[{ a_n \frac{x^n}{n!} }\right]} - 10 \sum^{\infty}_{n= 0}   {\left[{ \frac{x^n}{n!}+    }\right]} + 25 \sum^{\infty}_{n= 0 }   {\left[{  a_{n+  2 }\frac{x^n}{n!} }\right]}&= 0 \nonumber
\end{align}

By applying the definition of the exponential generating function at \eqref{eq:exp-gen-def-1} :

\begin{align}
    f''{\left({ x }\right)}- 10f'{\left({ x }\right)}+  25f{\left({ x }\right)}= 0 \nonumber \label{eq:rep-roots-func-ode}
\end{align}

By implementing the already well-established theory of linear ODE's, the characteristic equation for \eqref{eq:rep-roots-func-ode} can be expressed as:

\begin{align}
    m^2- 10m+  25 = 0 \nonumber \\
    {\left({ m- 5 }\right)}^2 = 0 \nonumber \\
    m= 5 \label{eq:rep-roots-recurrence-char-sol}
\end{align}

Herein lies a complexity, in order to solve this, the solution produced from \eqref{eq:rep-roots-recurrence-char-sol} can be used with the \emph{Reduction of Order} technique to produce a solution that will be of the form \textsection 4.3:

\begin{align}
    f{\left({ x }\right)}= c_1e^{5x} +  c_2 x e^{5x} \label{eq:rep-roots-ode-sol}
\end{align}

\eqref{eq:rep-roots-ode-sol} can be expressed in terms of the exponential power series in order to try and relate the solution for the function back to the generating function,
observe however the following power series identity (TODO Prove this):

\begin{align}
    x^ke^x &= \sum^{\infty}_{n= 0}   {\left[{ \frac{x^n}{{\left({ n- k }\right)}!} }\right]}, \quad \exists k \in \mathbb{Z}^+ \label{eq:uniq-roots-pow-series-ident}
\end{align}

by applying identity \eqref{eq:uniq-roots-pow-series-ident} to equation \eqref{eq:rep-roots-ode-sol}

\begin{align}
    \implies  f{\left({ x }\right)} &= \sum^{\infty}_{n= 0}   {\left[{ c_1 \frac{{\left({ 5x }\right)}^n}{n!} }\right]}  +  \sum^{\infty}_{n= 0}   {\left[{ c_2 n \frac{{\left({ 5x^n }\right)}}{n{\left({ n-1 }\right)}!} }\right]} \nonumber \\
 &= \sum^{\infty}_{n= 0}   {\left[{ \frac{x^n}{n!} {\left({ c_{1}5^n +  c_2 n 5^n   }\right)} }\right]} \nonumber
\end{align}

Given the defenition of the exponential generating function from \eqref{eq:exp-gen-def-1}

\begin{align}
    f{\left({ x }\right)}&=     \sum^{\infty}_{n= 0}   {\left[{ a_n \frac{x^n}{n!} }\right]} \nonumber \\
    \iff a_n &= c_{15}^n +  c_2n_5^n \nonumber \\ \nonumber
    \ \nonumber \\
    \square \nonumber
\end{align}
\subparagraph{Generalised Example}
\label{sec:org8dde82d}

\subparagraph{Proof}
\label{sec:org3b066c5}
In order to prove the the solution for a \(k^{\mathrm{th}}\) order recurrence relation with \(k\) repeated


Consider a recurrence relation of the form:

\begin{align}
     \sum^{k}_{n= 0}   {\left[{ c_i a_n }\right]}  = 0 \nonumber \\
      \implies  \sum^{\infty}_{n= 0}   \sum^{k}_{i= 0}   c_i a_n \frac{x^n}{n!} = 0 \nonumber \\
      \sum^{k}_{i= 0}   \sum^{\infty}_{n= 0}   c_i a_n \frac{x^n}{n!} \nonumber
\end{align}

By substituting for the value of the generating function (from \eqref{eq:exp-gen-def-1}):

\begin{align}
    \sum^{k}_{i= 0}   {\left[{ c_if^{{\left({ k }\right)}}  {\left({ x }\right)}    }\right]} \label{eq:gen-form-rep-roots-ode}
\end{align}

Assume that \eqref{eq:gen-form-rep-roots-ode} corresponds to a charecteristic polynomial with only 1 root of multiplicity \(k\), the solution would hence be of the form:

\begin{align}
			 & \sum^{k}_{i= 0}   {\left[{ c_i m^i }\right]} = 0 \wedge m=B, \enspace  \exists! B \in \mathbb{C} \nonumber \\
 \implies      f{\left({ x }\right)}&= \sum^{k}_{i= 0}   {\left[{ x^i A_i e^{mx} }\right]}, \quad \exists A \in \mathbb{C}^+, \enspace \forall i \in {\left[{ 1,k }\right]} \cap \mathbb{N}  \label{eq:sol-rep-roots-ode} \\
\end{align}

Recall the following power series identity (proved in section xxx):

\begin{align}
x^k e^x = \sum^{\infty}_{n= 0} {\left[{ \frac{x^n}{{\left({ n- k }\right)}!} }\right]}     \nonumber
\end{align}

By applying this to \eqref{eq:sol-rep-roots-ode} :

\begin{align}
f{\left({ x }\right)}&=     \sum^{k}_{i= 0}   {\left[{ A_i \sum^{\infty}_{n= 0}   {\left[{ \frac{{\left({ x m }\right)}^n}{{\left({ n- i }\right)}!} }\right]}  }\right]} \nonumber \\
&=     \sum^{\infty}_{n= 0}   {\left[{ \sum^{k}_{i=0} {\left[{ \frac{x^n}{n!}  \frac{n!}{{\left({ n- i }\right)}} A_i m^n }\right]}       }\right]} # \\
&=     \sum^{\infty}_{n= 0} {\left[{ \frac{x^n}{n!}   \sum^{k}_{i=0} {\left[{  \frac{n!}{{\left({ n- i }\right)}} A_i m^n }\right]}       }\right]}
\end{align}

Recall the generating function that was used to get \ref{eq:gen-form-rep-roots-ode}:

\begin{align}
f{\left({ x }\right)}&= \sum^{\infty}_{n= 0}   {\left[{ a_n \frac{x^n}{n!} }\right]}      \nonumber \\
 \implies  a_n &= \sum^{k}_{i= 0}   {\left[{ A_i \frac{n!}{{\left({ n- i }\right)}!} m^n  }\right]} \nonumber \\
 &= \sum^{k}_{i= 0}   {\left[{ m^n A_i \prod_{0}^{k} {\left[{ n- {\left({ i- 1 }\right)} }\right]}   }\right]}
& \intertext{$\because \enspace i \leq k$} \notag \\
 &= \sum^{k}_{i= 0} {\left[{ A_i^* m^n n^i }\right]}, \quad \exists A_i \in \mathbb{C}, \enspace \forall i\leqk \in \mathbb{Z}^+ \nonumber \\
\ \nonumber \\
\square \nonumber
\end{align}



\paragraph{General Proof}
\label{sec:org1ac5cfc}
In sections \ref{sec:org516b254} and \ref{sec:org516b254} it was shown that a recurrence relation can be related to an ODE and then that solution can be transformed to provide a solution for the recurrence relation, when the charecteristic polynomial has either complex roots or 1 repeated root. Generally the solution to a linear ODE will be a superposition of solutions for each root, repeated or unique and so here it will be shown that these two can be combined and that the solution will still hold.

Consider a Recursive relation with constant coefficients:

$$
\sum^{\infty}_{n= 0}   \left[ c_i \cdot  a_n \right] = 0, \quad \exists c \in
\mathbb{R}, \enspace \forall i<k\in\mathbb{Z}^+
$$

This can be expressed in terms of the exponential generating function:

$$
\sum^{\infty}_{n= 0}   \left[ c_i \cdot  a_n \right] = 0\\
\implies \sum^{\infty}_{n= 0}   \left[\sum^{\infty}_{n= 0}   \left[ c_i \cdot
a_n  \right]   \right] = 0
$$

\begin{itemize}
\item Use the Generating function to get an ODE
\item The ODE will have a solution that is a combination of the above two forms
\item The solution will translate back to a combination of both above forms
\end{itemize}



\subsection{Links to references}
\label{sec:org888beae}

\begin{enumerate}
\item \url{https://math.stackexchange.com/a/1775226}
\item \url{https://math.stackexchange.com/a/593553}
\item \url{https://www.maa.org/sites/default/files/pdf/upload\_library/22/Ford/IvanNiven.pdf}
\end{enumerate}

Misc

\begin{enumerate}
\item \url{https://brilliant.org/wiki/generating-functions-solving-recurrence-relations/}
\item \url{https://www.math.cmu.edu/\~af1p/Teaching/Combinatorics/Slides/Generating-Functions.pdf}
\item \url{https://www.math.cmu.edu/\~af1p/Teaching/Combinatorics/Slides/Generating-Functions.pdf}
\end{enumerate}

\section{Prove the Power Series}
\label{sec:org0a544af}
\subsection{General Power Series}
\label{sec:orgd7043b2}

\subsection{Exponential Power Series}
\label{sec:orgcb7b0a9}
\subsection{Extended Power Series (for Repeated Roots)}
\label{sec:orgf5900a9}

\subsection{Taylor Series}
\label{sec:org062633b}


When \(n\) is set to 0, the \emph{Taylor Series} reduces to the \emph{Mean Value Theorem}, when \(a\) is set to 0 the series is referred to as the \emph{Maclaurin Series}.
\end{document}
