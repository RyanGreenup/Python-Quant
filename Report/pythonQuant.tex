% Created 2020-08-10 Mon 22:54
% Intended LaTeX compiler: pdflatex
\documentclass[11pt]{article}
\usepackage[utf8]{inputenc}
\usepackage[T1]{fontenc}
\usepackage{graphicx}
\usepackage{grffile}
\usepackage{longtable}
\usepackage{wrapfig}
\usepackage{rotating}
\usepackage[normalem]{ulem}
\usepackage{amsmath}
\usepackage{textcomp}
\usepackage{amssymb}
\usepackage{capt-of}
\usepackage{hyperref}
\usepackage{minted}
\usepackage[citestyle=numeric, bibstyle=numeric,hyperref=true,backref=true, maxcitenames=3,url=true,backend=biber,natbib=true]{biblatex}
\addbibresource{../Resources/references.bib}
\AtEndDocument{\printbibliography}
\author{Ryan Greenup}
\date{\today}
\title{Python Quantitative Project}
\hypersetup{
 pdfauthor={Ryan Greenup},
 pdftitle={Python Quantitative Project},
 pdfkeywords={},
 pdfsubject={},
 pdfcreator={Emacs 26.3 (Org mode 9.4)}, 
 pdflang={English}}
\begin{document}

\maketitle
\tableofcontents




\begin{minted}[]{bash}
code /home/ryan/Dropbox/Studies/QuantProject/Current/Python-Quant/ & disown
\end{minted}

\section{What the unit involves.}
\label{sec:orgc40dc55}
Writing and presenting.
\section{Python}
\label{sec:orgc1442d0}
Two parts:
\begin{enumerate}
\item Figure out the python
\item IMPORTANT: MUST have new math
\begin{enumerate}
\item Dr. Hazrat may have some new math to use.
\end{enumerate}
\end{enumerate}

\section{Matrix Exponentiation}
\label{sec:org08b0575}
\subsubsection{Implementation in Sympy}
\label{sec:orgaaa51eb}
The Matrix Exponential is implemented in areas of:

\begin{itemize}
\item Graph Centrality modelling \cite{parkPowerWalkRevisiting2013}
\item Systems of Linear Differential Equations \cite[Ch. 8.4]{Zil2009}
\item Theory of Algebraic Lie Groups \cite[Ch. 2]{hallLieGroupsLie2015}
\end{itemize}

However the method to implement matrix exponentiation provided \href{https://docs.sympy.org/latest/tutorial/matrices.html}{by the documentation} \cite{MatricesSymPyDocumentation2020} and \href{https://github.com/sympy/sympy/issues/6218}{referenced in the development repository} \cite{MatrixExponentialIssue2019} does not appear to be implemented very well, for example the following provides a very long result:

\begin{minted}[]{python}
  from __future__ import division
  from sympy import *
  x, y, z, t = symbols('x y z t')
  k, m, n = symbols('k m n', integer=True)
  f, g, h = symbols('f g h', cls=Function)
  init_printing()
  init_printing(use_latex='mathjax', latex_mode='equation')


  import pyperclip
  def lx(expr):
      pyperclip.copy(latex(expr))
      print(expr)
\end{minted}

\begin{minted}[]{python}
A = Matrix([
    [11, 12, 13],
    [21, 22, 23],
    [31, 32, 33]
])

  expr = exp(A)
  expr.doit()
\end{minted}

$$
{
\scriptsize
\left[\begin{matrix}- \frac{1}{- \frac{33}{94} + \frac{5 \sqrt{1149}}{94}}
\left(\frac{- \frac{6552}{- \sqrt{1149} - 22} + 552}{\left(- \sqrt{1149} -
22\right) \left(- \frac{59 \sqrt{1149}}{95} - \frac{1837}{95}\right)} -
\frac{26}{- \sqrt{1149} - 22}\right) - \frac{1}{\frac{33 \sqrt{1149}}{2303} +
\frac{5745}{2303}} \left(-1 - \frac{1}{- \frac{33}{94} + \frac{5
\sqrt{1149}}{94}} \left(\frac{26}{- \sqrt{1149} - 22} - \frac{- \frac{6552}{-
\sqrt{1149} - 22} + 552}{\left(- \sqrt{1149} - 22\right) \left(- \frac{59
\sqrt{1149}}{95} - \frac{1837}{95}\right)} + 2\right)\right) \left(-
\frac{13}{-22 + \sqrt{1149}} + \frac{- \frac{3276}{-22 + \sqrt{1149}} +
276}{\left(-22 + \sqrt{1149}\right) \left(- \frac{1837}{95} + \frac{59
\sqrt{1149}}{95}\right)} - \frac{1}{- \frac{33}{94} + \frac{5 \sqrt{1149}}{94}}
\left(\frac{- \frac{3276}{- \sqrt{1149} - 22} + 276}{\left(- \sqrt{1149} -
22\right) \left(- \frac{59 \sqrt{1149}}{95} - \frac{1837}{95}\right)} -
\frac{13}{- \sqrt{1149} - 22}\right) \left(- \frac{26}{-22 + \sqrt{1149}} -
\frac{- \frac{273}{-22 + \sqrt{1149}} + 23}{- \frac{1837}{95} + \frac{59
\sqrt{1149}}{95}} + \frac{- \frac{6552}{-22 + \sqrt{1149}} + 552}{\left(-22 +
\sqrt{1149}\right) \left(- \frac{1837}{95} + \frac{59
\sqrt{1149}}{95}\right)}\right)\right) + \frac{1}{\left(\frac{33
\sqrt{1149}}{2303} + \frac{5745}{2303}\right) e^{-33 + \sqrt{1149}}} \left(-1 -
\frac{1}{- \frac{33}{94} + \frac{5 \sqrt{1149}}{94}} \left(\frac{26}{-
\sqrt{1149} - 22} - \frac{- \frac{6552}{- \sqrt{1149} - 22} + 552}{\left(-
\sqrt{1149} - 22\right) \left(- \frac{59 \sqrt{1149}}{95} -
\frac{1837}{95}\right)} + 2\right)\right) \left(- \frac{13}{-22 + \sqrt{1149}} +
\frac{- \frac{3276}{-22 + \sqrt{1149}} + 276}{\left(-22 + \sqrt{1149}\right)
\left(- \frac{1837}{95} + \frac{59 \sqrt{1149}}{95}\right)}\right) + 1 +
\left(\frac{- \frac{3276}{- \sqrt{1149} - 22} + 276}{\left(- \sqrt{1149} -
22\right) \left(- \frac{59 \sqrt{1149}}{95} - \frac{1837}{95}\right)} -
\frac{13}{- \sqrt{1149} - 22}\right) \left(- \frac{1}{\left(- \frac{33}{94} +
\frac{5 \sqrt{1149}}{94}\right) \left(\frac{33 \sqrt{1149}}{2303} +
\frac{5745}{2303}\right)} \left(-1 - \frac{1}{- \frac{33}{94} + \frac{5
\sqrt{1149}}{94}} \left(\frac{26}{- \sqrt{1149} - 22} - \frac{- \frac{6552}{-
\sqrt{1149} - 22} + 552}{\left(- \sqrt{1149} - 22\right) \left(- \frac{59
\sqrt{1149}}{95} - \frac{1837}{95}\right)} + 2\right)\right) \left(-
\frac{26}{-22 + \sqrt{1149}} - \frac{- \frac{273}{-22 + \sqrt{1149}} + 23}{-
\frac{1837}{95} + \frac{59 \sqrt{1149}}{95}} + \frac{- \frac{6552}{-22 +
\sqrt{1149}} + 552}{\left(-22 + \sqrt{1149}\right) \left(- \frac{1837}{95} +
\frac{59 \sqrt{1149}}{95}\right)}\right) + \frac{2}{- \frac{33}{94} + \frac{5
\sqrt{1149}}{94}}\right) e^{33 + \sqrt{1149}} & - \frac{1}{- \frac{33}{94} +
\frac{5 \sqrt{1149}}{94}} \left(\frac{- \frac{3276}{- \sqrt{1149} - 22} +
276}{\left(- \sqrt{1149} - 22\right) \left(- \frac{59 \sqrt{1149}}{95} -
\frac{1837}{95}\right)} - \frac{13}{- \sqrt{1149} - 22}\right) +
\frac{1}{\left(- \frac{33}{94} + \frac{5 \sqrt{1149}}{94}\right) \left(\frac{33
\sqrt{1149}}{2303} + \frac{5745}{2303}\right)} \left(- \frac{13}{-22 +
\sqrt{1149}} + \frac{- \frac{3276}{-22 + \sqrt{1149}} + 276}{\left(-22 +
\sqrt{1149}\right) \left(- \frac{1837}{95} + \frac{59 \sqrt{1149}}{95}\right)} -
\frac{1}{- \frac{33}{94} + \frac{5 \sqrt{1149}}{94}} \left(\frac{- \frac{3276}{-
\sqrt{1149} - 22} + 276}{\left(- \sqrt{1149} - 22\right) \left(- \frac{59
\sqrt{1149}}{95} - \frac{1837}{95}\right)} - \frac{13}{- \sqrt{1149} -
22}\right) \left(- \frac{26}{-22 + \sqrt{1149}} - \frac{- \frac{273}{-22 +
\sqrt{1149}} + 23}{- \frac{1837}{95} + \frac{59 \sqrt{1149}}{95}} + \frac{-
\frac{6552}{-22 + \sqrt{1149}} + 552}{\left(-22 + \sqrt{1149}\right) \left(-
\frac{1837}{95} + \frac{59 \sqrt{1149}}{95}\right)}\right)\right)
\left(\frac{13}{- \sqrt{1149} - 22} - \frac{- \frac{3276}{- \sqrt{1149} - 22} +
276}{\left(- \sqrt{1149} - 22\right) \left(- \frac{59 \sqrt{1149}}{95} -
\frac{1837}{95}\right)} + 1\right) - \frac{1}{\left(- \frac{33}{94} + \frac{5
\sqrt{1149}}{94}\right) \left(\frac{33 \sqrt{1149}}{2303} +
\frac{5745}{2303}\right) e^{-33 + \sqrt{1149}}} \left(- \frac{13}{-22 +
\sqrt{1149}} + \frac{- \frac{3276}{-22 + \sqrt{1149}} + 276}{\left(-22 +
\sqrt{1149}\right) \left(- \frac{1837}{95} + \frac{59
\sqrt{1149}}{95}\right)}\right) \left(\frac{13}{- \sqrt{1149} - 22} - \frac{-
\frac{3276}{- \sqrt{1149} - 22} + 276}{\left(- \sqrt{1149} - 22\right) \left(-
\frac{59 \sqrt{1149}}{95} - \frac{1837}{95}\right)} + 1\right) + \left(\frac{-
\frac{3276}{- \sqrt{1149} - 22} + 276}{\left(- \sqrt{1149} - 22\right) \left(-
\frac{59 \sqrt{1149}}{95} - \frac{1837}{95}\right)} - \frac{13}{- \sqrt{1149} -
22}\right) \left(\frac{1}{\left(- \frac{33}{94} + \frac{5
\sqrt{1149}}{94}\right)^{2} \left(\frac{33 \sqrt{1149}}{2303} +
\frac{5745}{2303}\right)} \left(- \frac{26}{-22 + \sqrt{1149}} - \frac{-
\frac{273}{-22 + \sqrt{1149}} + 23}{- \frac{1837}{95} + \frac{59
\sqrt{1149}}{95}} + \frac{- \frac{6552}{-22 + \sqrt{1149}} + 552}{\left(-22 +
\sqrt{1149}\right) \left(- \frac{1837}{95} + \frac{59
\sqrt{1149}}{95}\right)}\right) \left(\frac{13}{- \sqrt{1149} - 22} - \frac{-
\frac{3276}{- \sqrt{1149} - 22} + 276}{\left(- \sqrt{1149} - 22\right) \left(-
\frac{59 \sqrt{1149}}{95} - \frac{1837}{95}\right)} + 1\right) + \frac{1}{-
\frac{33}{94} + \frac{5 \sqrt{1149}}{94}}\right) e^{33 + \sqrt{1149}} &
\frac{1}{\left(\frac{33 \sqrt{1149}}{2303} + \frac{5745}{2303}\right) e^{-33 +
\sqrt{1149}}} \left(- \frac{13}{-22 + \sqrt{1149}} + \frac{- \frac{3276}{-22 +
\sqrt{1149}} + 276}{\left(-22 + \sqrt{1149}\right) \left(- \frac{1837}{95} +
\frac{59 \sqrt{1149}}{95}\right)}\right) - \frac{1}{\frac{33
\sqrt{1149}}{2303} + \frac{5745}{2303}} \left(- \frac{13}{-22 + \sqrt{1149}} +
\frac{- \frac{3276}{-22 + \sqrt{1149}} + 276}{\left(-22 + \sqrt{1149}\right)
\left(- \frac{1837}{95} + \frac{59 \sqrt{1149}}{95}\right)} - \frac{1}{-
\frac{33}{94} + \frac{5 \sqrt{1149}}{94}} \left(\frac{- \frac{3276}{-
\sqrt{1149} - 22} + 276}{\left(- \sqrt{1149} - 22\right) \left(- \frac{59
\sqrt{1149}}{95} - \frac{1837}{95}\right)} - \frac{13}{- \sqrt{1149} -
22}\right) \left(- \frac{26}{-22 + \sqrt{1149}} - \frac{- \frac{273}{-22 +
\sqrt{1149}} + 23}{- \frac{1837}{95} + \frac{59 \sqrt{1149}}{95}} + \frac{-
\frac{6552}{-22 + \sqrt{1149}} + 552}{\left(-22 + \sqrt{1149}\right) \left(-
\frac{1837}{95} + \frac{59 \sqrt{1149}}{95}\right)}\right)\right) -
\frac{e^{33 + \sqrt{1149}}}{\left(- \frac{33}{94} + \frac{5
\sqrt{1149}}{94}\right) \left(\frac{33 \sqrt{1149}}{2303} +
\frac{5745}{2303}\right)} \left(\frac{- \frac{3276}{- \sqrt{1149} - 22} +
276}{\left(- \sqrt{1149} - 22\right) \left(- \frac{59 \sqrt{1149}}{95} -
\frac{1837}{95}\right)} - \frac{13}{- \sqrt{1149} - 22}\right) \left(-
\frac{26}{-22 + \sqrt{1149}} - \frac{- \frac{273}{-22 + \sqrt{1149}} + 23}{-
\frac{1837}{95} + \frac{59 \sqrt{1149}}{95}} + \frac{- \frac{6552}{-22 +
\sqrt{1149}} + 552}{\left(-22 + \sqrt{1149}\right) \left(- \frac{1837}{95} +
\frac{59 \sqrt{1149}}{95}\right)}\right)\\-2 - \frac{1}{\left(-
\frac{1837}{95} + \frac{59 \sqrt{1149}}{95}\right) \left(\frac{33
\sqrt{1149}}{2303} + \frac{5745}{2303}\right) e^{-33 + \sqrt{1149}}} \left(-1 -
\frac{1}{- \frac{33}{94} + \frac{5 \sqrt{1149}}{94}} \left(\frac{26}{-
\sqrt{1149} - 22} - \frac{- \frac{6552}{- \sqrt{1149} - 22} + 552}{\left(-
\sqrt{1149} - 22\right) \left(- \frac{59 \sqrt{1149}}{95} -
\frac{1837}{95}\right)} + 2\right)\right) \left(- \frac{273}{-22 +
\sqrt{1149}} + 23\right) + \frac{2}{\frac{33 \sqrt{1149}}{2303} +
\frac{5745}{2303}} \left(-1 - \frac{1}{- \frac{33}{94} + \frac{5
\sqrt{1149}}{94}} \left(\frac{26}{- \sqrt{1149} - 22} - \frac{- \frac{6552}{-
\sqrt{1149} - 22} + 552}{\left(- \sqrt{1149} - 22\right) \left(- \frac{59
\sqrt{1149}}{95} - \frac{1837}{95}\right)} + 2\right)\right) \left(-
\frac{13}{-22 + \sqrt{1149}} + \frac{- \frac{3276}{-22 + \sqrt{1149}} +
276}{\left(-22 + \sqrt{1149}\right) \left(- \frac{1837}{95} + \frac{59
\sqrt{1149}}{95}\right)} - \frac{1}{- \frac{33}{94} + \frac{5 \sqrt{1149}}{94}}
\left(\frac{- \frac{3276}{- \sqrt{1149} - 22} + 276}{\left(- \sqrt{1149} -
22\right) \left(- \frac{59 \sqrt{1149}}{95} - \frac{1837}{95}\right)} -
\frac{13}{- \sqrt{1149} - 22}\right) \left(- \frac{26}{-22 + \sqrt{1149}} -
\frac{- \frac{273}{-22 + \sqrt{1149}} + 23}{- \frac{1837}{95} + \frac{59
\sqrt{1149}}{95}} + \frac{- \frac{6552}{-22 + \sqrt{1149}} + 552}{\left(-22 +
\sqrt{1149}\right) \left(- \frac{1837}{95} + \frac{59
\sqrt{1149}}{95}\right)}\right)\right) + \frac{1}{- \frac{33}{94} + \frac{5
\sqrt{1149}}{94}} \left(\frac{- \frac{13104}{- \sqrt{1149} - 22} + 1104}{\left(-
\sqrt{1149} - 22\right) \left(- \frac{59 \sqrt{1149}}{95} -
\frac{1837}{95}\right)} - \frac{52}{- \sqrt{1149} - 22}\right) - \frac{e^{33 +
\sqrt{1149}}}{- \frac{59 \sqrt{1149}}{95} - \frac{1837}{95}} \left(-
\frac{1}{\left(- \frac{33}{94} + \frac{5 \sqrt{1149}}{94}\right) \left(\frac{33
\sqrt{1149}}{2303} + \frac{5745}{2303}\right)} \left(-1 - \frac{1}{-
\frac{33}{94} + \frac{5 \sqrt{1149}}{94}} \left(\frac{26}{- \sqrt{1149} - 22} -
\frac{- \frac{6552}{- \sqrt{1149} - 22} + 552}{\left(- \sqrt{1149} - 22\right)
\left(- \frac{59 \sqrt{1149}}{95} - \frac{1837}{95}\right)} + 2\right)\right)
\left(- \frac{26}{-22 + \sqrt{1149}} - \frac{- \frac{273}{-22 + \sqrt{1149}} +
23}{- \frac{1837}{95} + \frac{59 \sqrt{1149}}{95}} + \frac{- \frac{6552}{-22 +
\sqrt{1149}} + 552}{\left(-22 + \sqrt{1149}\right) \left(- \frac{1837}{95} +
\frac{59 \sqrt{1149}}{95}\right)}\right) + \frac{2}{- \frac{33}{94} + \frac{5
\sqrt{1149}}{94}}\right) \left(- \frac{273}{- \sqrt{1149} - 22} + 23\right) &
\frac{\left(- \frac{273}{-22 + \sqrt{1149}} + 23\right) \left(\frac{13}{-
\sqrt{1149} - 22} - \frac{- \frac{3276}{- \sqrt{1149} - 22} + 276}{\left(-
\sqrt{1149} - 22\right) \left(- \frac{59 \sqrt{1149}}{95} -
\frac{1837}{95}\right)} + 1\right)}{\left(- \frac{1837}{95} + \frac{59
\sqrt{1149}}{95}\right) \left(- \frac{33}{94} + \frac{5 \sqrt{1149}}{94}\right)
\left(\frac{33 \sqrt{1149}}{2303} + \frac{5745}{2303}\right) e^{-33 +
\sqrt{1149}}} - \frac{2}{\left(- \frac{33}{94} + \frac{5 \sqrt{1149}}{94}\right)
\left(\frac{33 \sqrt{1149}}{2303} + \frac{5745}{2303}\right)} \left(-
\frac{13}{-22 + \sqrt{1149}} + \frac{- \frac{3276}{-22 + \sqrt{1149}} +
276}{\left(-22 + \sqrt{1149}\right) \left(- \frac{1837}{95} + \frac{59
\sqrt{1149}}{95}\right)} - \frac{1}{- \frac{33}{94} + \frac{5 \sqrt{1149}}{94}}
\left(\frac{- \frac{3276}{- \sqrt{1149} - 22} + 276}{\left(- \sqrt{1149} -
22\right) \left(- \frac{59 \sqrt{1149}}{95} - \frac{1837}{95}\right)} -
\frac{13}{- \sqrt{1149} - 22}\right) \left(- \frac{26}{-22 + \sqrt{1149}} -
\frac{- \frac{273}{-22 + \sqrt{1149}} + 23}{- \frac{1837}{95} + \frac{59
\sqrt{1149}}{95}} + \frac{- \frac{6552}{-22 + \sqrt{1149}} + 552}{\left(-22 +
\sqrt{1149}\right) \left(- \frac{1837}{95} + \frac{59
\sqrt{1149}}{95}\right)}\right)\right) \left(\frac{13}{- \sqrt{1149} - 22} -
\frac{- \frac{3276}{- \sqrt{1149} - 22} + 276}{\left(- \sqrt{1149} - 22\right)
\left(- \frac{59 \sqrt{1149}}{95} - \frac{1837}{95}\right)} + 1\right) +
\frac{1}{- \frac{33}{94} + \frac{5 \sqrt{1149}}{94}} \left(\frac{- \frac{6552}{-
\sqrt{1149} - 22} + 552}{\left(- \sqrt{1149} - 22\right) \left(- \frac{59
\sqrt{1149}}{95} - \frac{1837}{95}\right)} - \frac{26}{- \sqrt{1149} -
22}\right) - \frac{e^{33 + \sqrt{1149}}}{- \frac{59 \sqrt{1149}}{95} -
\frac{1837}{95}} \left(\frac{1}{\left(- \frac{33}{94} + \frac{5
\sqrt{1149}}{94}\right)^{2} \left(\frac{33 \sqrt{1149}}{2303} +
\frac{5745}{2303}\right)} \left(- \frac{26}{-22 + \sqrt{1149}} - \frac{-
\frac{273}{-22 + \sqrt{1149}} + 23}{- \frac{1837}{95} + \frac{59
\sqrt{1149}}{95}} + \frac{- \frac{6552}{-22 + \sqrt{1149}} + 552}{\left(-22 +
\sqrt{1149}\right) \left(- \frac{1837}{95} + \frac{59
\sqrt{1149}}{95}\right)}\right) \left(\frac{13}{- \sqrt{1149} - 22} - \frac{-
\frac{3276}{- \sqrt{1149} - 22} + 276}{\left(- \sqrt{1149} - 22\right) \left(-
\frac{59 \sqrt{1149}}{95} - \frac{1837}{95}\right)} + 1\right) + \frac{1}{-
\frac{33}{94} + \frac{5 \sqrt{1149}}{94}}\right) \left(- \frac{273}{-
\sqrt{1149} - 22} + 23\right) & \frac{1}{\frac{33 \sqrt{1149}}{2303} +
\frac{5745}{2303}} \left(- \frac{26}{-22 + \sqrt{1149}} + \frac{-
\frac{6552}{-22 + \sqrt{1149}} + 552}{\left(-22 + \sqrt{1149}\right) \left(-
\frac{1837}{95} + \frac{59 \sqrt{1149}}{95}\right)} - \frac{2}{- \frac{33}{94} +
\frac{5 \sqrt{1149}}{94}} \left(\frac{- \frac{3276}{- \sqrt{1149} - 22} +
276}{\left(- \sqrt{1149} - 22\right) \left(- \frac{59 \sqrt{1149}}{95} -
\frac{1837}{95}\right)} - \frac{13}{- \sqrt{1149} - 22}\right) \left(-
\frac{26}{-22 + \sqrt{1149}} - \frac{- \frac{273}{-22 + \sqrt{1149}} + 23}{-
\frac{1837}{95} + \frac{59 \sqrt{1149}}{95}} + \frac{- \frac{6552}{-22 +
\sqrt{1149}} + 552}{\left(-22 + \sqrt{1149}\right) \left(- \frac{1837}{95} +
\frac{59 \sqrt{1149}}{95}\right)}\right)\right) - \frac{- \frac{273}{-22 +
\sqrt{1149}} + 23}{\left(- \frac{1837}{95} + \frac{59 \sqrt{1149}}{95}\right)
\left(\frac{33 \sqrt{1149}}{2303} + \frac{5745}{2303}\right) e^{-33 +
\sqrt{1149}}} + \frac{e^{33 + \sqrt{1149}}}{\left(- \frac{33}{94} + \frac{5
\sqrt{1149}}{94}\right) \left(- \frac{59 \sqrt{1149}}{95} -
\frac{1837}{95}\right) \left(\frac{33 \sqrt{1149}}{2303} +
\frac{5745}{2303}\right)} \left(- \frac{273}{- \sqrt{1149} - 22} + 23\right)
\left(- \frac{26}{-22 + \sqrt{1149}} - \frac{- \frac{273}{-22 + \sqrt{1149}} +
23}{- \frac{1837}{95} + \frac{59 \sqrt{1149}}{95}} + \frac{- \frac{6552}{-22 +
\sqrt{1149}} + 552}{\left(-22 + \sqrt{1149}\right) \left(- \frac{1837}{95} +
\frac{59 \sqrt{1149}}{95}\right)}\right)\\- \frac{1}{- \frac{33}{94} + \frac{5
\sqrt{1149}}{94}} \left(\frac{- \frac{6552}{- \sqrt{1149} - 22} + 552}{\left(-
\sqrt{1149} - 22\right) \left(- \frac{59 \sqrt{1149}}{95} -
\frac{1837}{95}\right)} - \frac{26}{- \sqrt{1149} - 22}\right) -
\frac{1}{\frac{33 \sqrt{1149}}{2303} + \frac{5745}{2303}} \left(-1 - \frac{1}{-
\frac{33}{94} + \frac{5 \sqrt{1149}}{94}} \left(\frac{26}{- \sqrt{1149} - 22} -
\frac{- \frac{6552}{- \sqrt{1149} - 22} + 552}{\left(- \sqrt{1149} - 22\right)
\left(- \frac{59 \sqrt{1149}}{95} - \frac{1837}{95}\right)} + 2\right)\right)
\left(- \frac{13}{-22 + \sqrt{1149}} + \frac{- \frac{3276}{-22 + \sqrt{1149}} +
276}{\left(-22 + \sqrt{1149}\right) \left(- \frac{1837}{95} + \frac{59
\sqrt{1149}}{95}\right)} - \frac{1}{- \frac{33}{94} + \frac{5 \sqrt{1149}}{94}}
\left(\frac{- \frac{3276}{- \sqrt{1149} - 22} + 276}{\left(- \sqrt{1149} -
22\right) \left(- \frac{59 \sqrt{1149}}{95} - \frac{1837}{95}\right)} -
\frac{13}{- \sqrt{1149} - 22}\right) \left(- \frac{26}{-22 + \sqrt{1149}} -
\frac{- \frac{273}{-22 + \sqrt{1149}} + 23}{- \frac{1837}{95} + \frac{59
\sqrt{1149}}{95}} + \frac{- \frac{6552}{-22 + \sqrt{1149}} + 552}{\left(-22 +
\sqrt{1149}\right) \left(- \frac{1837}{95} + \frac{59
\sqrt{1149}}{95}\right)}\right)\right) + \frac{1}{\left(\frac{33
\sqrt{1149}}{2303} + \frac{5745}{2303}\right) e^{-33 + \sqrt{1149}}} \left(-1 -
\frac{1}{- \frac{33}{94} + \frac{5 \sqrt{1149}}{94}} \left(\frac{26}{-
\sqrt{1149} - 22} - \frac{- \frac{6552}{- \sqrt{1149} - 22} + 552}{\left(-
\sqrt{1149} - 22\right) \left(- \frac{59 \sqrt{1149}}{95} -
\frac{1837}{95}\right)} + 2\right)\right) + 1 + \left(- \frac{1}{\left(-
\frac{33}{94} + \frac{5 \sqrt{1149}}{94}\right) \left(\frac{33
\sqrt{1149}}{2303} + \frac{5745}{2303}\right)} \left(-1 - \frac{1}{-
\frac{33}{94} + \frac{5 \sqrt{1149}}{94}} \left(\frac{26}{- \sqrt{1149} - 22} -
\frac{- \frac{6552}{- \sqrt{1149} - 22} + 552}{\left(- \sqrt{1149} - 22\right)
\left(- \frac{59 \sqrt{1149}}{95} - \frac{1837}{95}\right)} + 2\right)\right)
\left(- \frac{26}{-22 + \sqrt{1149}} - \frac{- \frac{273}{-22 + \sqrt{1149}} +
23}{- \frac{1837}{95} + \frac{59 \sqrt{1149}}{95}} + \frac{- \frac{6552}{-22 +
\sqrt{1149}} + 552}{\left(-22 + \sqrt{1149}\right) \left(- \frac{1837}{95} +
\frac{59 \sqrt{1149}}{95}\right)}\right) + \frac{2}{- \frac{33}{94} + \frac{5
\sqrt{1149}}{94}}\right) e^{33 + \sqrt{1149}} & - \frac{1}{- \frac{33}{94} +
\frac{5 \sqrt{1149}}{94}} \left(\frac{- \frac{3276}{- \sqrt{1149} - 22} +
276}{\left(- \sqrt{1149} - 22\right) \left(- \frac{59 \sqrt{1149}}{95} -
\frac{1837}{95}\right)} - \frac{13}{- \sqrt{1149} - 22}\right) +
\frac{1}{\left(- \frac{33}{94} + \frac{5 \sqrt{1149}}{94}\right) \left(\frac{33
\sqrt{1149}}{2303} + \frac{5745}{2303}\right)} \left(- \frac{13}{-22 +
\sqrt{1149}} + \frac{- \frac{3276}{-22 + \sqrt{1149}} + 276}{\left(-22 +
\sqrt{1149}\right) \left(- \frac{1837}{95} + \frac{59 \sqrt{1149}}{95}\right)} -
\frac{1}{- \frac{33}{94} + \frac{5 \sqrt{1149}}{94}} \left(\frac{- \frac{3276}{-
\sqrt{1149} - 22} + 276}{\left(- \sqrt{1149} - 22\right) \left(- \frac{59
\sqrt{1149}}{95} - \frac{1837}{95}\right)} - \frac{13}{- \sqrt{1149} -
22}\right) \left(- \frac{26}{-22 + \sqrt{1149}} - \frac{- \frac{273}{-22 +
\sqrt{1149}} + 23}{- \frac{1837}{95} + \frac{59 \sqrt{1149}}{95}} + \frac{-
\frac{6552}{-22 + \sqrt{1149}} + 552}{\left(-22 + \sqrt{1149}\right) \left(-
\frac{1837}{95} + \frac{59 \sqrt{1149}}{95}\right)}\right)\right)
\left(\frac{13}{- \sqrt{1149} - 22} - \frac{- \frac{3276}{- \sqrt{1149} - 22} +
276}{\left(- \sqrt{1149} - 22\right) \left(- \frac{59 \sqrt{1149}}{95} -
\frac{1837}{95}\right)} + 1\right) - \frac{\frac{13}{- \sqrt{1149} - 22} -
\frac{- \frac{3276}{- \sqrt{1149} - 22} + 276}{\left(- \sqrt{1149} - 22\right)
\left(- \frac{59 \sqrt{1149}}{95} - \frac{1837}{95}\right)} + 1}{\left(-
\frac{33}{94} + \frac{5 \sqrt{1149}}{94}\right) \left(\frac{33
\sqrt{1149}}{2303} + \frac{5745}{2303}\right) e^{-33 + \sqrt{1149}}} +
\left(\frac{1}{\left(- \frac{33}{94} + \frac{5 \sqrt{1149}}{94}\right)^{2}
\left(\frac{33 \sqrt{1149}}{2303} + \frac{5745}{2303}\right)} \left(-
\frac{26}{-22 + \sqrt{1149}} - \frac{- \frac{273}{-22 + \sqrt{1149}} + 23}{-
\frac{1837}{95} + \frac{59 \sqrt{1149}}{95}} + \frac{- \frac{6552}{-22 +
\sqrt{1149}} + 552}{\left(-22 + \sqrt{1149}\right) \left(- \frac{1837}{95} +
\frac{59 \sqrt{1149}}{95}\right)}\right) \left(\frac{13}{- \sqrt{1149} - 22} -
\frac{- \frac{3276}{- \sqrt{1149} - 22} + 276}{\left(- \sqrt{1149} - 22\right)
\left(- \frac{59 \sqrt{1149}}{95} - \frac{1837}{95}\right)} + 1\right) +
\frac{1}{- \frac{33}{94} + \frac{5 \sqrt{1149}}{94}}\right) e^{33 + \sqrt{1149}}
& \frac{1}{\left(\frac{33 \sqrt{1149}}{2303} + \frac{5745}{2303}\right) e^{-33 +
\sqrt{1149}}} - \frac{1}{\frac{33 \sqrt{1149}}{2303} + \frac{5745}{2303}}
\left(- \frac{13}{-22 + \sqrt{1149}} + \frac{- \frac{3276}{-22 + \sqrt{1149}} +
276}{\left(-22 + \sqrt{1149}\right) \left(- \frac{1837}{95} + \frac{59
\sqrt{1149}}{95}\right)} - \frac{1}{- \frac{33}{94} + \frac{5 \sqrt{1149}}{94}}
\left(\frac{- \frac{3276}{- \sqrt{1149} - 22} + 276}{\left(- \sqrt{1149} -
22\right) \left(- \frac{59 \sqrt{1149}}{95} - \frac{1837}{95}\right)} -
\frac{13}{- \sqrt{1149} - 22}\right) \left(- \frac{26}{-22 + \sqrt{1149}} -
\frac{- \frac{273}{-22 + \sqrt{1149}} + 23}{- \frac{1837}{95} + \frac{59
\sqrt{1149}}{95}} + \frac{- \frac{6552}{-22 + \sqrt{1149}} + 552}{\left(-22 +
\sqrt{1149}\right) \left(- \frac{1837}{95} + \frac{59
\sqrt{1149}}{95}\right)}\right)\right) - \frac{e^{33 + \sqrt{1149}}}{\left(-
\frac{33}{94} + \frac{5 \sqrt{1149}}{94}\right) \left(\frac{33
\sqrt{1149}}{2303} + \frac{5745}{2303}\right)} \left(- \frac{26}{-22 +
\sqrt{1149}} - \frac{- \frac{273}{-22 + \sqrt{1149}} + 23}{- \frac{1837}{95} +
\frac{59 \sqrt{1149}}{95}} + \frac{- \frac{6552}{-22 + \sqrt{1149}} +
552}{\left(-22 + \sqrt{1149}\right) \left(- \frac{1837}{95} + \frac{59
\sqrt{1149}}{95}\right)}\right)\end{matrix}\right]
}
$$
Simplifying this result doesn't seem to help either:

\begin{minted}[]{python}
simplify(expr)
\end{minted}

$$
{
\scriptsize
\left[\begin{matrix}\frac{1}{12 \left(-1065889 + 33298 \sqrt{1149}\right)
e^{\sqrt{1149}}} \left(- 8625947 e^{33 + 2 \sqrt{1149}} - 2131778
e^{\sqrt{1149}} - 2032943 e^{33} + 74651 \sqrt{1149} e^{33} + 66596 \sqrt{1149}
e^{\sqrt{1149}} + 258329 \sqrt{1149} e^{33 + 2 \sqrt{1149}}\right) & \frac{1}{6
\left(-1065889 + 33298 \sqrt{1149}\right) e^{\sqrt{1149}}} \left(- 965995
e^{33 + 2 \sqrt{1149}} - 66596 \sqrt{1149} e^{\sqrt{1149}} - 1165783 e^{33} +
36081 \sqrt{1149} e^{33} + 2131778 e^{\sqrt{1149}} + 30515 \sqrt{1149} e^{33 + 2
\sqrt{1149}}\right) & \frac{1}{6 \left(-43187463 + 1274291 \sqrt{1149}\right)
e^{\sqrt{1149}}} \left(- 2723224 \sqrt{1149} e^{33 + 2 \sqrt{1149}} - 43187463
e^{\sqrt{1149}} - 49129419 e^{33} + 1448933 \sqrt{1149} e^{33} + 1274291
\sqrt{1149} e^{\sqrt{1149}} + 92316882 e^{33 + 2 \sqrt{1149}}\right)\\\frac{1}{6
\left(-1065889 + 33298 \sqrt{1149}\right) e^{\sqrt{1149}}} \left(- 66949 e^{33 +
2 \sqrt{1149}} - 66596 \sqrt{1149} e^{\sqrt{1149}} - 2064829 e^{33} + 61128
\sqrt{1149} e^{33} + 2131778 e^{\sqrt{1149}} + 5468 \sqrt{1149} e^{33 + 2
\sqrt{1149}}\right) & \frac{1}{3 \left(4213 + 125 \sqrt{1149}\right)
e^{\sqrt{1149}}} \left(44 e^{33} + 2 \sqrt{1149} e^{33} + 8426 e^{\sqrt{1149}} +
250 \sqrt{1149} e^{\sqrt{1149}} + 4169 e^{33 + 2 \sqrt{1149}} + 123 \sqrt{1149}
e^{33 + 2 \sqrt{1149}}\right) & \frac{1}{6 \left(-43187463 + 1274291
\sqrt{1149}\right) e^{\sqrt{1149}}} \left(- 78841 \sqrt{1149} e^{33 + 2
\sqrt{1149}} - 2548582 \sqrt{1149} e^{\sqrt{1149}} - 89061939 e^{33} + 2627423
\sqrt{1149} e^{33} + 86374926 e^{\sqrt{1149}} + 2687013 e^{33 + 2
\sqrt{1149}}\right)\\\frac{1}{12 \left(-1065889 + 33298 \sqrt{1149}\right)
e^{\sqrt{1149}}} \left(- 236457 \sqrt{1149} e^{33 + 2 \sqrt{1149}} - 6226373
e^{33} - 2131778 e^{\sqrt{1149}} + 66596 \sqrt{1149} e^{\sqrt{1149}} + 169861
\sqrt{1149} e^{33} + 8358151 e^{33 + 2 \sqrt{1149}}\right) & \frac{1}{6
\left(-1065889 + 33298 \sqrt{1149}\right) e^{\sqrt{1149}}} \left(- 25145
\sqrt{1149} e^{33 + 2 \sqrt{1149}} - 66596 \sqrt{1149} e^{\sqrt{1149}} - 3163663
e^{33} + 91741 \sqrt{1149} e^{33} + 2131778 e^{\sqrt{1149}} + 1031885 e^{33 + 2
\sqrt{1149}}\right) & \frac{1}{6 \left(-43187463 + 1274291 \sqrt{1149}\right)
e^{\sqrt{1149}}} \left(- 86942856 e^{33 + 2 \sqrt{1149}} - 128994459 e^{33} -
43187463 e^{\sqrt{1149}} + 1274291 \sqrt{1149} e^{\sqrt{1149}} + 3805913
\sqrt{1149} e^{33} + 2565542 \sqrt{1149} e^{33 + 2
\sqrt{1149}}\right)\end{matrix}\right]
}
$$

Methods suggested online only provide numerical solutions or partial sums:

\begin{itemize}
\item \href{https://stackoverflow.com/questions/47240208/sympy-symbolic-matrix-exponential}{python -
Sympy Symbolic Matrix Exponential - Stack Overflow}
\item \href{https://stackoverflow.com/a/50718831/12843551}{python - Exponentiate
symbolic matrix expression using SymPy - Stack Overflow}
\item \href{https://stackoverflow.com/a/54025116/12843551}{Calculate state transition
matrix in python - Stack Overflow}
\end{itemize}

Instead this will need to be implemented from first principles.

\subsection{Theory}
\label{sec:org0c07e01}
\subsubsection{Matrix Exponentiation}
\label{sec:orgb0a793e}
A Matrix Exponential is defined by using the ordinary exponential power series \cite[Ch. 2]{hallLieGroupsLie2015},\cite[Ch. 8.4]{Zil2009} (should we prove the power series generally?):

\begin{align}
    e^{\mathit{\mathbf{X}}} = \sum^{\infty}_{k= 0}   \left[ \frac{1}{k!} \cdot  \mathit{\mathbf{X^k}} \right] 
\end{align}

This definition can be expanded upon however by using properties of logarithms:

\begin{align}
    b &= e^{\log_e{\left( b \right) }}, \quad \forall b \in \mathbb{C} \label{eq:bydef}\\
 \implies  b^{\mathbf{X}}&= \left( e^{\log_e{\left( b \right) }} \right)^{\mathbf{X}} \label{eq:tojustify} \\
  \implies  b^{\mathbf{X}} &= e^{\log_e{b}  \mathbf{X} }
\end{align}

The identity in \eqref{eq:bydef} is justified by the definition of the complex log. However some discussion is required for \eqref{eq:tojustify}  because it is not clear that the
exponential will generally distribute throught he parenthesis like so \(\left( a\cdot b \right)^{k} = a^k\cdot b^k\), for example
consider \(\left( \left[ - 1 \right]^2 \cdot 3
\right)^{\frac{1}{2}} \neq \left[ - 1 \right]^{\frac{2}{2}} \cdot
3^{\frac{1}{2}}\).

A sufficient condition for this identity is \(k \in
\mathbb{Z}^{*}\), consider this example which will be important later:

\begin{align}
    \left( \log_e{\left( b \right) }\mathbf{X} \right)^{k} , \quad \forall k \in \mathbb{Z^{*}}
\end{align}

Because multiplication is commutative \(\forall z \in \mathbb{C}\), this could be
re-expressed in the form:

\begin{align}
 \left( \log_e{\left( b \right) }\mathbf{X} \right)^{k} &=    \underbrace{\log_e{\left( b \right) }\cdot  \log_e{\left( b \right) } \cdot  \log_e{\left( b \right) }\ldots }_{k \text{ times}} \times \underbrace{\mathbf{X}\mathbf{X}\mathbf{X}\ldots}_{k \text{ times}} \notag \\
 &= \log_e^k{\left( b \right) } \mathbf{X}^k \label{eq:matpower}
\end{align}

Now consider the the following by applying \eqref{eq:matpower}:

\begin{align}
    e^{X}&= \sum^{\infty}_{k= 0}   \left[ \frac{1}{k!} \mathbf{X}^{k} \right]  \notag \\
    \implies  e^{bX}&= \sum^{\infty}_{k= 0}   \left[ \frac{1}{k!} \left( b\mathbf{X} \right)^{k} \right] \quad \forall b \in \mathbb{C} \notag \\
    &= \sum^{\infty}_{k= 0}   \left[ \frac{1}{k!}b^k \mathbf{X}^k \right] \notag \\
    &= \left( e^b \right)^{\mathbf{X}} \notag \\
    &\implies  e^{b \mathbf{X}} = e^{\mathbf{X}b}= \left( e^b \right)^{\mathbf{X}}= \left( e^{\mathbf{X}} \right)^b  \qquad \qquad \square \label{eq:expmatpower}
\end{align}

So the matrix exponential for an arbitrary base could be given by:

\begin{align}
   b = e^{\log_e{\left( b \right) }}, \quad \forall b \in \mathbb{C} \notag \\
    \implies  b^{\mathbf{X}} &= \left( e^{\log_e{\left( b \right) }} \right)^{\mathbf{X}} \notag \\
     \text{as per \eqref{eq:expmatpower}} \notag \\
    b^{\mathbf{X}} &=  e^{\log_e{\left( b \right) } {\mathbf{X}}}  \notag \\
     b^{\mathbf{X}} &= \sum^{\infty}_{k= 0}   \left[ \frac{\left( \log_e{\left( b \right) }\mathbf{X} \right)^k}{k!} \right]  \notag \\
     &= \sum^{\infty}_{k= 0}   \left[ \frac{\log_e ^{k}{\left( b \right) }}{k!}\mathbf{X}^{k} \right]
\end{align}

This is also consistent with the \emph{McLaurin Series} expansion of \(b^{\mathbf{X}}
\enspace (\forall b \in \mathbb{C})\):

\begin{align*}
f\left( x \right) &= \sum^{\infty}_{k= 0}   \left[ \frac{f^{\left( n \right)}\left( 0 \right)}{k!} x^{k} \right]  \\
\implies  b^x &= \sum^{\infty}_{k= 0}  \left[ \frac{\frac{\mathrm{d}^n }{\mathrm{d} x^n}\left( b^x \right) \vert_{x=0}   }{k!} x^k \right]  \\
\implies  b^{\mathbf{X}} &= \sum^{\infty}_{k= 0}   \left[ \frac{\frac{\mathrm{d}^n }{\mathrm{d}\mathbf{X}^n  } \left( b^{\mathbf{X}} \right) \vert_{\mathbf{X}= \mathbf{O}}}{k!} \mathbf{X}^k \right]
\end{align*}

By ordinary calculus identities we have\(f\left( x \right) = b^{x} \implies
f^{\left( n \right)}\left( x \right) = b^{x} \log_e^n{\left( b \right)}\) which
distribute through a matrix and hence:

\begin{align*}
    b^x &= \sum^{\infty}_{k= 0}  \left[ \frac{b^0 \log_e^k{\left( b \right) }}{k!} x^k \right]  \\
    \implies  b^{\mathbf{X}} &= \sum^{\infty}_{k= 0}  \left[ \frac{b^0 \log_e^k{\left( b \right) }}{k!} \mathbf{X}^k \right]
\end{align*}

By the previous identity:

\begin{align*}
\implies  b^{\mathbf{X}} &= \sum^{\infty}_{k= 0}  {\left[ \frac{{\left( \log_e{\left( b \right) } \mathbf{X} \right)}^k}{k!} \right]} \\
    &= e^{\log_e{\left( b \right) } \mathbf{X}}
\end{align*}

\subsubsection{Matrix-Matrix Exponentiation}
\label{sec:org6389c99}

Matrix-Matrix exponentiation has applications in quantum mechanics \cite[p. 84]{barradasIteratedExponentiationMatrixMatrix1994}.

As for Matrices with the requirements:

\begin{enumerate}
\item Square
\item Normal:
\begin{itemize}
\item Commutes with it's congugate transpose
\end{itemize}
\item Non Singular
\item Non Zero Determinant
\end{enumerate}

\begin{align*}
    \left| \left| A-I \right| \right|<1  &\implies  e^{\log_e{\left( \mathbf{A} \right) }} = \mathbf{A} \enspace \text{(By Lie Groups Springer Textbook)}\\
                     &\implies  \mathbf{A}^{\mathbf{B}} =\left( e^{\log_e{\left( \mathbf{A} \right) }} \right)^{\mathbf{B}} \\
             & \text{Similar justification as \eqref{eq:expmatpower}} \\
             & \implies  \mathbf{A}^{\mathbf{B}}= e^{\log_e{\left( \mathbf{A} \right) } \mathbf{B}}
\end{align*}

However the following identities are by \textbf{Definition} anyway: \cite{barradasIteratedExponentiationMatrixMatrix1994}

\begin{align}
\mathbf{A}^{\mathbf{B}}&= e^{\log_e{\left( \mathbf{A} \right) } \mathbf{B}} \\
\ ^{\mathbf{B}} \mathbf{A} &= e^{ \mathbf{B} \log_e{\left( \mathbf{A} \right) } }
\end{align}

\subsection{An alternative Implementation in Sympy}
\label{sec:orgc7506d3}

\begin{minted}[]{python}
def matexp(mat, base = E):
      """
      Return the Matrix Exponential of a square matrix
      """
      import copy
      import sympy
  # Should realy test for sympy vs numpy array
  # Test for Square Matrix
      if mat.shape[0] != mat.shape[1]:
          print("ERROR: Only defined for Square matrices")
          return
      m = zeros(mat.shape[0])
      for i in range(m.shape[0]):
          for j in range(m.shape[1]):
              m[i,j] = Sum((mat[i,j]*ln(base))**k/factorial(k), (k, 0, oo)).doit()
      return m
\end{minted}

\begin{minted}[]{python}
matexp(A, pi)
\end{minted}

$$
\left[\begin{matrix}\pi^{11} & \pi^{12} & \pi^{13}\\\pi^{21} & \pi^{22} &
\pi^{23}\\\pi^{31} & \pi^{32} & \pi^{33}\end{matrix}\right]
$$

But it would be nice to expand this to matrix bases for there uses in quantum
mechanics.

The built in method for a**mat is not implemented.

there is exp(mat) but this returns garbage (see github issue), (see other
solution on stack exchange that is numeric and example)

show our method with proofs of

cauchy power taylor then exp

then show our code

\begin{minted}[]{python}
A = Matrix([ [11,12,13], [21,22,23], [31,32,33] ])

  B = Matrix([
      [1,2,3],
      [4,5,6],
      [7,8,9]
  ])


  A**B
\end{minted}
\section{Recursive Relations}
\label{sec:orga125b82}
A recursive relation is of the form:

$$
\sum^{\infty}_{n= 0}   \left[ c_i \cdot  a_n \right] = 0, \quad \exists c \in
\mathbb{R}, \enspace \forall i<k\in\mathbb{Z}^+
$$

In order to find a solution for \(a_n\), the following one-to-one
correspondence can be used to relate the vector space of the sequence to the
power series ring:(cite stackExchange[1]):

$$\begin{aligned}
g: \left( a_n \right)_{n\in\mathbb{Z}^+} \rightarrow \mathbb{C}\left[ \left[ x \right]  \right]: g\left( a_n \right) = \sum^{\infty}_{n= 0}\left[ \frac{x^n}{n!} a_n \right] 
\end{aligned}$$

This technique is referred to as generating functions.
\cite{lehmanReadingsMathematicsComputer2010}


\subsection{{\bfseries\sffamily TODO} Generating Functions}
\label{sec:org0dcad4d}
\subsubsection{{\bfseries\sffamily TODO} Example}
\label{sec:orga66fb02}
\subsection{{\bfseries\sffamily TODO} Exponential Generating Function}
\label{sec:org3d2247c}
\subsubsection{{\bfseries\sffamily TODO} Example}
\label{sec:orgeed800d}
\subsubsection{{\bfseries\sffamily TODO} Homogoneous Proof}
\label{sec:orgc586201}
\begin{enumerate}
\item {\bfseries\sffamily TODO} Derivative of Exponential Generating Function
\label{sec:org90d342b}
CITE OPEN MIT COURSEWARE SLIDES
\item {\bfseries\sffamily TODO} Unique Roots of Characteristic Equation
\label{sec:org25a8b07}
\begin{enumerate}
\item Example
\label{sec:orgf6cf24d}
\item Proof
\label{sec:org7343115}
\end{enumerate}
\item {\bfseries\sffamily TODO} Repeated Roots of Characteristic Equation
\label{sec:org9f4e42e}
\begin{enumerate}
\item Example
\label{sec:org239bacf}
\item Proof
\label{sec:org0b2c4be}
\end{enumerate}

\item {\bfseries\sffamily TODO} General Proof
\label{sec:orgdd581ff}
Consider a Recursive relation with constant coefficients:

$$
\sum^{\infty}_{n= 0}   \left[ c_i \cdot  a_n \right] = 0, \quad \exists c \in
\mathbb{R}, \enspace \forall i<k\in\mathbb{Z}^+
$$

This can be expressed in terms of the exponential generating function:

$$
\sum^{\infty}_{n= 0}   \left[ c_i \cdot  a_n \right] = 0\\
\implies \sum^{\infty}_{n= 0}   \left[\sum^{\infty}_{n= 0}   \left[ c_i \cdot
a_n  \right]   \right] = 0
$$
\end{enumerate}

\subsection{Pandoc Conversion}
\label{sec:orgd3f3c89}
Given the Linear Recurrence Relation:

\begin{align*}
a_0= 1 \\
a_0= 1 \\
a_{n+  2} =  a_{n+  1 +  2 a_n}, \quad n \geq 0
\end{align*}

To solve this we can use what's known as a
\href{https://en.wikipedia.org/wiki/Generating\_function}{Generating
Function}, \hyperref[sec:orgdfa6c3d]{see the disucssion below}

We will make consider the function \(f(x)\) as shown below in:

\begin{align}
f\left( x \right)= \sum^{\infty}_{n= 0}   \left[ a_nx^n \right] \label{eq:pow-gen-func-np0}
\end{align}


It can be shown (see \eqref{iterate-pow-gen-func}) that:


\begin{align}
    \sum^{\infty}_{n= 0}  \left[ a_{n+  1} x^n \right] &= \frac{f\left( x \right)- a_0}{x} \label{eq:pow-gen-func-np1} \\
\sum^{\infty}_{n= 0}  \left[ a_{n+  2} x^n \right]  &= \frac{f\left( x \right) - a_0 - a_1x}{x^2} \label{eq:pow-gen-func-np2}
\end{align}

So to use the generating Function consider:

\begin{align}
    2a_n +  a_{n+  1 }&= a_{n+  2} \nonumber \\
    2a_nx^n +  a_{n+  1 } x^n &= a_{n+  2} x^n \nonumber \\
    \sum^{\infty}_{n= 0}   \left[ 2a_nx^n \right] + \sum^{\infty}_{n= 0}   \left[  a_{n+  1 } x^n  \right]   &= \sum^{\infty}_{n= 0}   \left[ a_{n+  2} x^n   \right] \label{eq:series-rep-pow-example}
\end{align}

Observe that in \eqref{eq:series-rep-pow-exampluse} tuse te

By applying the previous identity shown in \eqref{eq:pow-gen-func-np0}, \eqref{eq:pow-gen-func-np1} and \eqref{eq:pow-gen-func-np2}:

\begin{align}
2f\left( x \right) +  \frac{f\left( x \right)- a_0}{x} &= \frac{f\left( x \right)- a_0}{- a_1x}x^2 \nonumber \\
\implies  f\left( x \right) &=  \frac{1}{1- x- x^2} \label{eq:power-series-form-example}
\end{align}

\begin{center}
\begin{tabular}{l}
WARNING\\
\hline
I accidently dropped the \(2\) here, it doesn't matter but it does show that how this could be dealt with algebraically\\
\end{tabular}
\end{center}

The function \(f(x)\) in \eqref{eq:power-series-form-example} can be solved by way of a power series, ( see for example \href{./University/Analysis/11\_Series.md}{11\textsubscript{Series}}), but first it is
necessary to use partial fractions to split it up:


By partial fractions it is known:

\begin{align*}
    f\left( x \right)&= \frac{1}{1- x- x^2}\\
&= \frac{- 1}{x^2 +  x -  1}\\
&= \frac{- 1}{\left( x- 2 \right)\left( x- 1 \right)}\\
&= \frac{A_1}{x- 2}+  \frac{A_2}{x- 1}, \quad A_i \in \mathbb{R}, i \in \mathbb{Z}^+ \\
 \implies  - 1 &= A_1\left( x- 1 \right) +  A_2\left( x- 2 \right)\\
 \text{Let $x$ = 2:}\\
 - 1&= A_1\left( 2-1 \right) +  0 \\
&= A_1 = - 1 \\
 \text{Let $x$ = 1:}\\
 - 1 &=  0 +  A_2 \left( 1- 2 \right) \\
 \implies  A_2&= 1 \\
 \text{Hence:}\\
 f\left( x \right)&=    \frac{1}{x- 1} - \frac{1}{x- 2}
\end{align*}

Now because it is known that:

\begin{align*}
\sum^{\infty}_{n= 0}\left[ rx^n \right] = \frac{1}{1- rx^n}
\end{align*}

we can conclude that:

\begin{align*}
\frac{1}{x- 1}&= -\frac{1}{1 -\left( 1 \right) x} \\
&= -\sum^{\infty}_{n= 0}\left[ x^n \right]  \\
\frac{-1}{x- 2} &= \frac{1}{2- x} \\
&= \frac{1}{2}\frac{1}{1-\frac{1}{2}x} \\
&= \frac{1}{2} \sum^{\infty}_{n= 0}\left[ \left( \frac{1}{2}x \right) ^n \right]
\end{align*}

and so:

\begin{align*}
f\left( x \right) &= \frac{1}{2}\sum^{\infty}_{n= 0}\left[ \left( \frac{1}{2}x \right) ^n \right] - \sum^{\infty}_{n= 0}\left[ x^n \right] \\
f\left( x \right) &= \sum^{\infty}_{n= 0}\left[ \frac{1}{2}\left( \frac{1}{2}x \right) ^n -x^n \right]  \\
f\left( x \right) &= \sum^{\infty}_{n= 0}\left[ \frac{1}{2 \cdot 2^n} x^n -x^n \right]  \\
f\left( x \right) &= \sum^{\infty}_{n= 0}\left[x^n {\left( {\frac{1}{2 \cdot 2^n} -1} \right) } \right]  \\
 \implies  a_n &= \frac{1}{2 \cdot 2^n} - 1
\end{align*}

\subsubsection{Generating Functions}
\label{sec:orgdfa6c3d}
A \href{https://en.wikipedia.org/wiki/Generating\_function}{Generating
Function} is a way of encoding an
\href{https://en.wikipedia.org/wiki/Infinite\_sequence}{infinite series} of
numbers (\(a_n\)) by treating them as the coeficcients of a power series
(\(\sum^\infty_{n = 0} \left[ a_nx^n \right]\)).

The variable remains in an indeterminate form and they were first
introduced by Abraham De Moivre in 1730 in order to solve the general
linear recurrence problem [\textsuperscript{1}]

[\textsuperscript{1}]: Donald E. Knuth, The Art of Computer Programming, Volume 1
Fundamental Algorithms (Third Edition) Addison-Wesley. ISBN
0-201-89683-4. Section 1.2.9: ``Generating Functions''.

\subsubsection{Using the Power series for the Exponential Function}
\label{sec:orgbcf1a0b}
\begin{enumerate}
\item Motivation
\label{sec:org3874f38}
Consider the \emph{Fibonacci Sequence}:


\begin{align*}
    a_{n}&= a_{n - 1} + a_{n - 2}\\
\iff a_{n+  2} &= a_{n+  1} +  a_n
\end{align*}


Solving this outright is quite difficult, previously we used a power
series generating function to solve it, something to the effect of:


\begin{align*}
x^2 f\left( x \right) -  x f \left( x \right) -  f\left( x \right)=  0
\end{align*}


This however is still a little tricky, however, just from observation,
the following would be fairly easy to deal with:


\begin{align*}
f''\left( x \right)- f'\left( x \right)- f\left( x \right)=  0
\end{align*}


This would however imply that \(f\left( x \right)= e^x\) because
\(\frac{\mathrm{d}\left( e^x \right) }{\mathrm{d} x} = e^x\), but that's
fine because we have a power series for that already:


\begin{align*}
f\left( x \right)= e^rx = \sum^{\infty}_{n= 0}   \left[ r \frac{x^n}{n!} \right]
\end{align*}


So this would give an easy means by which to solve the linear recurrence
relation.

\item Solving the Sequence
\label{sec:org9275af1}
Now this is all well and good but if we could relate this to \(f(x)=e^x\)
we'd really be cooking with fire because we could connect linear
recurrence relations to non-homogenous linear differential equations.

Consider using the following generating function:


\begin{align*}
    f\left( x \right) &=  \sum^{\infty}_{n= 0}   \left[ a_{n} \cdot  \frac{x^n}{n!} \right]   &= e^x \\
    \text{ $\mathcal{TODO::}$ The real trick is showing this derivative property } \\
    f'\left( x \right) &=  \sum^{\infty}_{n= 0}   \left[ a_{n+1} \cdot  \frac{x^n}{n!} \right]  &= e^x \\
    f''\left( x \right) &=  \sum^{\infty}_{n= 0}   \left[ a_{n+2} \cdot  \frac{x^n}{n!} \right] &= e^x \\
\end{align*}


So the recursive relation from above could be expressed:


\begin{align*}
a_{n+  2}    &= a_{n+  1} +  a_{n}\\
\frac{x^n}{n!}   a_{n+  2}    &= \frac{x^n}{n!}\left( a_{n+  1} +  a_{n}  \right)\\
\sum^{\infty}_{n= 0} \left[ \frac{x^n}{n!}   a_{n+  2} \right]        &= \sum^{\infty}_{n= 0}   \left[ \frac{x^n}{n!} a_{n+  1} \right]  + \sum^{\infty}_{n= 0}   \left[ \frac{x^n}{n!} a_{n}  \right]  \\
f''\left( x \right) &= f'\left( x \right)+  f\left( x \right)
\end{align*}


Using the theory of higher order linear differential equations with
constant coefficients it can be shown:


\begin{align*}
f\left( x \right)= c_1 \cdot  \mathrm{exp}\left[ \left( \frac{1- \sqrt{5} }{2} \right)x \right] +  c_2 \cdot  \mathrm{exp}\left[ \left( \frac{1 +  \sqrt{5} }{2} \right) \right]
\end{align*}


By equating this to the power series:


\begin{align*}
f\left( x \right)&= \sum^{\infty}_{n= 0}   \left[ \left( c_1\left( \frac{1- \sqrt{5} }{2} \right)^n +  c_2 \cdot  \left( \frac{1+ \sqrt{5} }{2} \right)^n \right) \cdot  \frac{x^n}{n} \right]
\end{align*}


Now given that:


\begin{align*}
f\left( x \right)= \sum^{\infty}_{n= 0}   \left[ a_n \frac{x^n}{n!} \right]
\end{align*}


We can conclude that:


\begin{align*}
a_n = c_1\cdot  \left( \frac{1- \sqrt{5} }{2} \right)^n +  c_2 \cdot  \left( \frac{1+  \sqrt{5} }{2} \right)
\end{align*}


By applying the initial conditions:


\begin{align*}
a_0= c_1 +  c_2  \implies  c_1= - c_2\\
a_1= c_1 \left( \frac{1+ \sqrt{5} }{2} \right) -  c_1 \frac{1-\sqrt{5} }{2}  \implies  c_1 = \frac{1}{\sqrt{5} }
\end{align*}


And so finally we have:


\begin{align*}
    a_n &= \frac{1}{\sqrt{5} } \left[ \left( \frac{1+  \sqrt{5} }{2}  \right)^n -  \left( \frac{1- \sqrt{5} }{2} \right)^n \right] \\
&= \frac{\varphi^n - \psi^n}{\sqrt{5} } \\
&= \frac{\varphi^n -  \psi^n}{\varphi - \psi}
\end{align*}


where:

\begin{itemize}
\item \(\varphi = \frac{1+ \sqrt{5} }{2} \approx 1.61\ldots\)
\item \(\psi = 1-\varphi = \frac{1- \sqrt{5} }{2} \approx 0.61\ldots\)
\end{itemize}

Open Questions:

\begin{itemize}
\item Show that the derivitive of the power series is \(a_{n+ 2}\)
\item Redo the initial problem for the Fibonacci Sequence
\item Extend this to a non-homogenous equation
\item Extend this to all linear recursion problems with first order ODES
\item Show that this is an isomorphism Lindear ODEs with constant
coefficients to recursive relations with constant coefficients.
\end{itemize}
\end{enumerate}

\subsubsection{References}
\label{sec:orgba0dc62}
\begin{enumerate}
\item \url{https://math.stackexchange.com/a/1775226}
\item \url{https://math.stackexchange.com/a/593553}
\item \url{https://www.maa.org/sites/default/files/pdf/upload\_library/22/Ford/IvanNiven.pdf}
\end{enumerate}

Misc

\begin{enumerate}
\item \url{https://brilliant.org/wiki/generating-functions-solving-recurrence-relations/}
\item \url{https://www.math.cmu.edu/\~af1p/Teaching/Combinatorics/Slides/Generating-Functions.pdf}
\item \url{https://www.math.cmu.edu/\~af1p/Teaching/Combinatorics/Slides/Generating-Functions.pdf}
\end{enumerate}
\end{document}
