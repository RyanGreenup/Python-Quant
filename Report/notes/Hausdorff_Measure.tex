%% ! TEX root = /home/ryan/Dropbox/Studies/MathModelling/Notes/Diffeq/latex_working/00_master.tex
% Copy realpath with Spc-f-y

\documentclass[12 pt]{article}
\usepackage{/home/ryan/Dropbox/profiles/Templates/LaTeX/ScreenStyle}

\input{./Dimension_Diagram_Styles.tikzstyles}
\usepackage{tikzit}

%\usepackage{~/Dropbox/profiles/Templates/LaTeX/ScreenStyle}
%LaTeX doesn't work with the ~


%%%%%%%Title
\title{\color{coltit} \Huge Notes on the HaussDorf Measure and Dimension}
\author{Ryan Greenup ; 1780-5315}



%%%%%%%%%%%%%%%%%%%%%4%%%%%%%%%
%%%%%%%Document%%%%%%%%%%%%%%%%
%%%%%%%%%%%%%%%%%%%%%%%% %%%%%%%
\begin{document}

\maketitle
\tableofcontents

%\begin{tikzpicture}
	\begin{pgfonlayer}{nodelayer}
		\node [style=none] (0) at (3.5, 6.25) {};
		\node [style=none] (1) at (2, 4.5) {};
		\node [style=none] (2) at (1.5, 3.5) {};
		\node [style=none] (3) at (1.25, 1.75) {};
		\node [style=none] (4) at (4.25, 4.5) {};
		\node [style=none] (5) at (3.5, 2.75) {};
		\node [style=none] (6) at (5, 7) {};
		\node [style=none] (7) at (4.25, 5.25) {};
		\node [style=none] (8) at (5.5, 2.75) {};
		\node [style=none] (9) at (4.75, 1) {};
		\node [style=none] (10) at (7.5, 5.75) {};
		\node [style=none] (11) at (6.75, 4) {};
		\node [style=none] (12) at (5.75, 8) {};
		\node [style=none] (13) at (5.75, 8) {$\mathbb{R}$};
		\node [style=none] (14) at (6.5, 7) {};
		\node [style=none] (15) at (0, 0) {};
		\node [style=none] (16) at (0, 12) {};
		\node [style=none] (17) at (12, 0) {};
	\end{pgfonlayer}
	\begin{pgfonlayer}{edgelayer}
		\draw [style=new edge style 0] (1.center)
			 to [in=255, out=30, looseness=1.75] (0.center)
			 to [in=45, out=180, looseness=3.00] cycle;
		\draw [style=new edge style 0] (3.center)
			 to [in=255, out=-30, looseness=1.75] (2.center)
			 to [bend right, looseness=0.75] cycle;
		\draw [style=new edge style 0] (5.center)
			 to [bend left=45] (4.center)
			 to [bend right=60, looseness=1.50] cycle;
		\draw [style=new edge style 0] (7.center)
			 to [bend right=135, looseness=2.75] (6.center)
			 to [in=120, out=-180, looseness=1.50] cycle;
		\draw [style=new edge style 0] (9.center)
			 to [bend right=105, looseness=1.25] (8.center)
			 to [in=45, out=-180, looseness=2.50] cycle;
		\draw [style=new edge style 0] (11.center)
			 to [in=255, out=30, looseness=1.75] (10.center)
			 to [in=45, out=-180, looseness=1.50] cycle;
		\draw [style=ArrowLeft] (16.center) to (15.center);
		\draw [style=ArrowRight] (15.center) to (17.center);
	\end{pgfonlayer}
\end{tikzpicture}


For small values of $s$ (i.e. less than the dimension of  $F$), the value of $\mathcal{H}^s$  will be $\infty$.

Consider some value $s$ such that the Hausdorff measure is not infinite, i.e. values of $s$: 

\footnote{Could fractal dimensions be complex? Maybe there could be a proof to show that the dimension is necessarily complex.}

$$
\begin{aligned}
    \mathcal{H}^s = L \in \mathbb{R}
\end{aligned}
$$

Consider a dimensional value $t$ that is larger than  $s$ and observe that:
$$
\begin{aligned}
0<s<t  \implies   \sum_{i}  \left[ \left\levert U_i \right\rvert^t \right] &= \sum_{i}\left[ \left\levert U_i \right\rvert^{t- s} \cdot  \left\lvert U_i \right\rvert^s \right] \\
									   &\leq \sum_{i} \left[ \delta^{t - s} \cdot \left\lvert U_i \right\rvert^s  \right]    \\
									   &= \delta^{t- s}\sum_{i}   \left[ \left\lvert U_i \right\rvert^s \right] 									   \\
									   
\end{aligned}
$$

Now if $\lim_{\delta \rightarrow 0}\left[ \sum_{i}   \left\lvert U_i \right\rvert^s \right]$ is defined as a non-infinite value:

$$
\begin{aligned}
    \lim_{\delta \rightarrow 0} \left( \sum_{i}   \left[ \left\lvert U_i \right\rvert^t \right]  \right) & \leq \lim_{\delta}\left( \delta^{t- s} \sum_{i}   \left[ \left\lvert U_i \right\rvert^s \right]  \right) \\
													 &\leq \lim_{\delta \rightarrow 0}\left( \delta^{t - s} \right) \cdot  \lim_{\delta \rightarrow 0}\left( \sum_{i} \left[ \left\lvert U_i \right\rvert^s \right]    \right) \\
													 &\leq 0
\end{aligned}
$$

Thus for $\forall t > s$, if $s$ is a finite limit value:

\begin{align}
    s \in \mathbb{R}  \implies  \mathcal{H}^t\left( F \right)= 0 \label{eq:hdfzero}
\end{align}

hence the value of the hausdorff measure, across various dimension values of $s$ is only a finite, non-zero value, when $s = \mathrm{dim}\left( F \right)$ this is visualised in figure [[fig:hausdorff-vals]].

\end{document}



%%% Local Variables: 
%%% mode: latex
%%% TeX-master: "/home/ryan/Dropbox/Studies/MathModelling/Notes/Diffeq/latex_working/00_master.tex"
%%% End: 

